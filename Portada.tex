\chapter*{Resumen}

Este documento presenta detalladamente el proceso de desarrollo de una extensi�n para un Sistema de Gesti�n de Aprendizaje (SGA) que permite el soporte de encuentros presenciales o seminarios entre estudiantes e instructores. 

El m�dulo apoya a tres tipos de usuarios en la gesti�n de los encuentros. Los administradores, que crean los cursos y asignan los instructores. Los instructores que pueden administrar la asistencia y calificaci�n; por �ltimo, los estudiantes que seleccionan cursos a los que asistir.

Para la realizaci�n de este m�dulo se us� la metodolog�a de desarrollo de software �gil e iterativo por fases SCRUM. Para el desarrollo se utiliz� el conjunto de soluciones inform�ticas WAMP, que consiste en Windows como plataforma de sistema operativo, Apache para el servicio web, SQL Server con gestor de bases de datos y el lenguaje multiprop�sito PHP.

Se logr� mediante este proyecto que el SGA base de la compa��a contenga la nueva funcionalidad que permite ser extendida y personalizada dependiendo de las necesidades de distintos clientes. Adem�s, el m�dulo se implant� con �xito en la arquitectura de un cliente que previamente pose�a el servicio de SGA.

Palabras clave: SGA, PHP, desarrollo, seminario.
