\chapter*{Agradecimientos}

A mis padres y mi tía Evelin, por ser siempre un impulso y un apoyo que me ha permitido llegar a lugares a los que nunca me hubiese imaginado, gracias a ustedes puedo decir que nunca me ha faltado nada, sobre todo amor.

Al resto de la familia por propiciar el mejor ambiente para mi desarrollo, en ustedes constantemente encuentro ejemplos a seguir, consejos, cariño y una inmensa responsabilidad de recordar de donde vengo con orgullo compartir los valores que de ustedes y con ustedes aprendí. 

A Clara Cangiano, te separo aquí de la familia, pero solo para que sepas que a ella perteneces. Gracias por compartir toda tu vida conmigo, de ti he aprendido mucho. El amor sin una cesta llena haz como que no lo has visto. Siempre mantuviste la cesta llena. No creas que olvidaré la promesa que te hice.

A mis amigos mas cercanos, cada uno con un aporte muy especial: Guillermo Hernández, con el que siempre fue bueno competir, pero al mismo tiempo me recordó ser humilde, no te quedes atrás. Luis Colorado, una de esas personas que a veces no te crees que existen, su entrega y positividad es única. Rafael Delgado, siempre estas allí viejo, nos veremos pronto. Mailyn Guevara por lo fácil que es compartir todo contigo y Fabiana Abdallah, gracias a tí valoro la sinceridad y entiendo que se puede conseguir.

A la gente del Laboratorio Docente de Aulas Computarizadas (MAC). No solo por ser mi casa en Caracas, si no por hacer de esa casa un hogar. Compartir con ustedes cada día fue una experiencia única de aprendizaje y superación sobre la computación y la vida. La mejor decisión que he tomado es hacer esa admisión.

A todas las demás personas que hacen vida en la Universidad Simón Bolívar, junto y gracias a ustedes no solo me formé como ingeniero, dadas las circunstancias, aprendí que hay que luchar, que la vida no es fácil, junto a ustedes estuve luchando en todos los frentes codo a codo y el que se roza con diamantes de alguna forma se pule.

Luis Colorado y Andrés Navarro, gracias por hacer los últimos pasos no solo amenos, sino posibles, que viva el tridente.