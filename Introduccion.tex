\chapter*{Introducción}
\thispagestyle{empty} % Quitar el número
\addcontentsline{toc}{chapter}{Introducción} \markboth{INTRODUCCIÓN}{}
El presente documento describe el proceso de desarrollo del módulo de encuentros presenciales para un Sistema de Gestión de Aprendizaje (SGA) para Fischer Knoblauch \& Co (FKC) como proyecto de pasantías realizado por su autor.

\section*{Antecedentes}
\addcontentsline{toc}{section}{Antecedentes}
El SGA básico de FKC es un producto que soporta las funciones básicas de cualquier sistema de gestión aprendizaje simple. Como lo son: manejo de usuarios y contenidos; segimiento del proceso de aprendizaje, evaluaciones y herramientas de comunicación como foros y mensajes privados. Este sistema es personalizado e instalado generalmente en la infraestructura del cliente según sus necesidades.

\section*{Justificación e importancia}
\addcontentsline{toc}{section}{Justificación e importancia}
Entre las funciones convencionales de los SGA actuales se encuentra el manejo de encuentros presenciales, funcionalidad considerada como necesaria por actuales y potenciales clientes. Siendo estos encuentros de importancia crítica para el flujo del conocimiento de mediana a alta complejidad que no puede ser expresado fácilmente por medio de cursos en línea. Por lo tanto, es de interés para la compañía poseer esta funcionalidad en su sistema.


\section*{Planteamiento del problema}
\addcontentsline{toc}{section}{Planteamiento del problema}
Se identificó la necesidad de extender el sistema de FKC con un módulo que le permita a sus clientes gestionar encuentros presenciales entre instructores y estudiantes. Al mismo tiempo este módulo debía adaptarse a la estructura de cursos y grupos previamente existente en el sistema, trantando de incluir la menor complejidad posible para su futura extensión y personalización.

Para lograr esto, es necesario un profundo entendimiento del producto, con el fin de desarrollar un módulo que mantenga el mismo estilo tanto en la programación, como en el funcionamiento y la interfaz. 


\section*{Objetivos}
\addcontentsline{toc}{section}{Objetivos}
A continuación, se exponen los objetivos generales y específicos que se buscan alcanzar en este desarrollo con la finalidad de contextualizar al lector respecto al informe de este proyecto de pasantía.

\subsection*{Objetivo general}
\addcontentsline{toc}{subsection}{Objetivo general}
El objetivo general de este proyecto es desarrollar un módulo que permita extender el sistema de gestión de aprendizaje de FKC y lo acerque a contener las funciones convencionales de los sistemas en la actualidad y que al mismo tiempo posea la flexibilidad de ser personalizado para las distintas necesidades de los clientes. Que se ofrezca también la opción de configurar un SGA sin incluir el módulo en cuestión.

\subsection*{Objetivos específicos}
\addcontentsline{toc}{subsection}{Objetivos específicos}

\begin{itemize}
	\item Extender la base de datos para el soporte del módulo.
	\item Desarrollar un módulo que permita el manejo de las locaciones donde se imparten los seminarios, manejado por el administrador.
	\item Integrar el nuevo tipo de curso a las funcionalidades previas de administrador sobre otros tipos de cursos, como estadísticas y asignación a grupos.
	\item Crear un nuevo tipo de usuario, instructor, que tenga potestad sobre los seminarios.
	\item Integrar el nuevo tipo de curso en la interfaz del estudiante del sistema.
	\item Integrar el módulo tanto en el sistema base de FKC como en Bibliomed, uno de los clientes que posee un SGA.
\end{itemize}








