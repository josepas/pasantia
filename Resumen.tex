\chapter*{Resumen}

Este documento presenta detalladamente el proceso de desarrollo de una extensión para un \gls{SGA} que permite el soporte de encuentros presenciales o seminarios entre estudiantes e instructores. 

El módulo apoya a tres tipos de usuarios en la gestión de los encuentros. Los administradores, que crean los cursos y asignan los instructores. Los instructores que pueden administrar la asistencia y calificación; por último, los estudiantes que seleccionan cursos a los que asistir.

Para la realización de este módulo se usó la metodología de desarrollo de software ágil e iterativo por fases SCRUM. Para el desarrollo se utilizó el conjunto de soluciones informáticas \gls{WAMP}, que consiste en Windows como plataforma de sistema operativo, Apache para el servicio web, \gls{SQL} Server con gestor de bases de datos y el lenguaje multipropósito PHP.

Se logró mediante este proyecto que el SGA base de la compañía contenga la nueva funcionalidad que permite ser extendida y personalizada dependiendo de las necesidades de distintos clientes. Además, el módulo se implantó con éxito en la arquitectura de un cliente que previamente poseía el servicio de \gls{SGA}.

Palabras clave: \gls{SGA}, \gls{PHP}, desarrollo, seminario.




