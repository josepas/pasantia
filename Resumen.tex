\chapter*{Resumen}

Este documento presenta detalladamente el proceso de desarrollo de una extensión para un SGA (Sistema de Gestión de Aprendizaje) que permite el soporte de encuentros presenciales o seminarios entre estudiantes e instructores. 

El módulo apoya a tres tipos de usuarios en la gestión de los encuentros. Los administradores, que crean los cursos y asignan los instructores. Los instructores que pueden administrar la asistencia y calificación, por último los estudiantes que seleccionan cursos a los que asistir.

Para la realización de este módulo se uso la metodología de desarrollo de software ágil e iterativo por fases SCRUM. Para el desarrollo se usó el conjunto de soluciones informaticas WAMP, que consiste en Windows como plataforma del sistema operativo, Apache para el servicio web, SQL Server con gestor de bases de datos y el lenguaje multipropósito PHP.

Se logró mediante este proyecto que el SGA base de la compañia contenga la nueva funcionalidad que permite ser extendendida y personalizada dependiendo de las necesidades de distintos clientes. Además, el módulo se implantó con éxito en la arquitectura de un cliente que previamente poseía el servicio de SGA.

Palabras clave: SGA, PHP, ágil, seminario.




