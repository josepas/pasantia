\chapter*{Conclusiones y Recomendaciones}
\thispagestyle{empty} % Quitar el número
\addcontentsline{toc}{chapter}{Conclusiones y Recomendaciones} \markboth{CONCLUSIONES Y RECOMENDACIONES}{}

La implementación e implantación de las funcionalidades de este proyecto de pasantías concluyó exitosamente y en el tiempo estipulado, el pasante se integró satisfactoriamente al equipo de programación de la empresa FKC con los que compartió durante el desarrollo. Derivando en que módulos para el manejo de seminarios; presenciales y en línea, así como un módulo para el soporte de las ubicaciones donde estos seminarios se realizan fueran agregados al SGA base de FKC y al SGA de Bibliomed, uno de los clientes de la empresa. Permitiendo a los usuarios de dichos sistemas más flexibilidad a la hora de idear planes de estudios dentro del sistema, integrando soporte para el \emph{b-learning}.

El proyecto fue realizado bajo la metodología SCRUM, la cual permitió que su desarrollo fuese planificado, eficiente y organizado, llevando a cabo los cambios necesarios para adaptarse a los nuevos requerimientos que constantemente fueron surgiendo a lo largo del mismo. El proyecto estuvo comprendido por diez sprints en donde se fueron desarrollando las historias de usuario que surgieron del levantamiento inicial de los requerimientos. Ordenadas en un marco de desarrollo de abajo hacia arriba donde el pasante intentó construir primero las funcionalidades más pequeñas, para luego construir con ellas las más grandes.

El desarrollo de los módulos se realizó bajo el patrón de arquitectura MVC, con la meta de mantener una estructura clara dentro del código que pueda ser mantenida y extendida por futuros desarrolladores. Durante la implementación del proyecto de pasantía, se utilizaron herramientas como el lenguaje de programación PHP, SQL y JavaScript para lograr los objetivos planteados.

Algunas recomendaciones surgen de la interacción del pasante con el SGA producido por FKC.

Se recomienda el uso de un manejador de versiones por parte del equipo de programación. Los beneficios podrían ser incalculables para la empresa pues este tipo de herramientas además de ser casi de uso obligado en el desarrollo de software desde hace más de quince años traen consigo una serie de bondades difíciles de rechazar, más aún si se toma en cuenta el \emph{workflow} de FKC. Entre estas:

\begin{itemize}
	\item Manejo de distintos ambientes de trabajo (desarrollo, producción, prueba) y sus respectivas versiones que pueden ser compartidas por todos los desarrolladores de la empresa.
	
	\item Mejor gestión del equipo, ofrece una interfaz centralizada para entender quien hizo que, cuándo y cómo.

	\item Facilita la construcción de mejores estimaciones para el desarrollo de los proyectos.

	\item Una importante reducción del tiempo en la implantación del sistema comparándola con el servicio de archivos manualmente a través de FTP.

	\item Soporte de la base de código externo a la empresa, haciéndola independiente del \emph{hardware} interno.

	\item Eliminar la práctica de realización de \emph{bakups} antes de cualquier cambio importante al sistema que consumen tiempo de los desarrolladores y espacio de disco en los servidores.

	\item Independencia de las bases de código de desarrollo y producción, donde desarrollo nuevo para el sistema no obstruya la implantación de un sistema nuevo o requiera el trabajo en una copia distinta de la base de código.
\end{itemize}

Si a todos estos beneficios le agregamos que FKC es una empresa que sirve a por lo menos quince otras, estos beneficios se multiplican, dado que existen distintas bases de código, repartidas en varios servidores a los que constantemente se realizan actualizaciones. Entre estas herramientas se encuentran GIT, Subversion, entre otras.

Se recomienda el uso de manejo de dependencias dentro del lenguaje de programación en los distintos proyectos de la empresa para el correcto manejo de los ambientes de desarrollo, entre estas herramientas se encuentran \emph{composer} para PHP y npm. 

En tiempos actuales, donde la globalización impera el pasante considera importante el desarrollo del sistema completamente en el idioma inglés. De esta forma se amplía fácilmente la cantidad de personas empleables por la empresa e incluso facilita la implementación de personal de apoyo fuera de las fronteras alemanas, dados los altos salarios que se necesitan para mantener un programador en dicho país.

Se recomienda una separación de los roles de diseñador y programador, haciendo uso eficiente del departamento de diseño gráfico dentro de la empresa. El pasante considera una buena práctica generar primero un diseño de como lucirá la interfaz y como serán las interacciones, para que luego la funcionalidad sea plasmada por el programador. Mezclar estos dos trabajos crea fases de espera entre el desarrollo de lo que el programador cree es una interfaz válida y la revisión por parte del departamento gráfico, especialmente cuando no están claros los requerimientos inicialmente.

Refiriéndose a la base de código el pasante ofrece las siguientes recomendaciones:

El largo de muchos de los archivos y de muchas de las funciones dentro del sistema es excesivo. Esto complica a los programadores trabajar en conjunto en un mismo proyecto, así como la legibilidad del código y aumentando la cohesión del mismo. De esta forma es difícil que un pequeño cambio no afecte alguna otra funcionalidad. Es además una de las primeras pistas que evidencian la mezcla de la lógica del negocio con el rol de los modelos y la interfaz definidos en el patrón MVC. Todo esto hace que la curva de aprendizaje de un nuevo desarrollador sea aún más empinada.

El código en su mayoría no es reusable, pedazos de código que realizan la misma función se ven en distintos archivos, replicando conocimiento y haciendo imposible la mantenibilidad eficiente del sistema. Se recomienda el uso del patrón MVC o en su defecto de cualquier otro patrón que se considere oportuno luego de un análisis de las funcionalidades del sistema. Además, se recomienda el uso de patrones de diseño, que favorezca la estructuración del código, promoviendo así la reusabilidad y facilitando el trabajo en equipo y las revisiones por parte del jefe de proyectos.

La lógica de negocio se encuentra mezclada con el acceso a los modelos y a la interfaz, haciendo imposible la asignación de desarrolladores a distintas tareas y eliminando la oportunidad de integrar a un experto en interfaces al equipo de desarrollo del sistema, es decir, cambiar la interfaz es casi imposible sin cambiar como se acceden a los datos o interferir en las reglas del negocio. Recordando que la personalización visual del sistema para cada cliente es una de las características más importantes para FKC.

La implementación de pruebas a lo largo del sistema. El sistema no contiene pruebas que lo sustenten. No tenerlas trae consigo un conjunto de problemas en el desarrollo de software. Al crear una nueva funcionalidad es conveniente ejecutar las pruebas para asegurarse que ninguna otra funcionalidad es afectada, esto combinado con los puntos antes descritos es caldo de cultivo para la proliferación de errores dentro del sistema. La realización de pruebas puede consumir tiempo al inicio del desarrollo, pero a medida que el sistema crece la retribución de ese tiempo invertido crece exponencialmente, pues elimina el tiempo de prueba que se consume luego de cada cambio importante, por ejemplo, un cambio de versión de PHP, entre otros.

Eliminar el uso de \emph{iframes} HTML para el desarrollo de la interfaz, esta es una técnica ampliamente criticada pues dificulta el manejo de la navegación dentro del sistema y no aprovecha el uso de elementos ya cargados en las páginas donde estos se llaman. Limitando su uso a la demostración de contenido multimedia u otras páginas fuera del sistema, para lo que estos fueron construidos.

Promover el uso de peticiones del tipo AJAX (menos recargas completas de las páginas dentro del sistema), implementación de APIs internas e implementar manejo de \emph{cache} para las consultas a la base de datos, reduciendo así el trabajo de los servidores web y de bases de datos o el futuro costo de \emph{hosting} si se decide implementar soluciones en la nube. Aumentando la cantidad de clientes atendidos con las capacidades actuales de la empresa.

Por último, el sistema no posee documentación alguna, exceptuando las autogeneradas por algunas de las herramientas y no se invierte tiempo en mantenerlas. Esto combinado con la estructura del código complica el desarrollo de software.

El pasante por lo tanto recomienda una reestructuración y reescritura completa del SGA base de FKC. Que bien podría seguir siendo en PHP o podría explorar algunas tecnologías nuevas. Que pueda aprovechar las bondades de un \emph{framework} para PHP como Laravel o Symphony. Estos frameworks facilitan la construcción de una estructura que permite el reúso de código, el trabajo en equipo y recortan los tiempos de desarrollo. Estas cualidades son muy importantes para FKC, ya que constantemente se encuentran desarrollando individualmente para algunos de sus clientes y poder portar las funcionalidades realizadas de un cliente a otro puede liberar tiempo para la creación de nuevas funcionalidades.

Para esta reestructuración se recomienda el uso de una metodología de desarrollo de software, que permita un desarrollo planificado, eficiente y organizado. que no descuide la realización de pruebas de software automatizadas y tenga como pilar fundamental el trabajo colaborativo basado en una herramienta de control de versiones correctamente usada.


