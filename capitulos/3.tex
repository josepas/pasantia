\chapter{Marco Tecnológico}
\thispagestyle{empty} % Quitar el número

A continuación, se presentan los aspectos tecnológicos relacionados con el desarrollo del proyecto de pasantía, se describen brevemente y se indica su uso como parte de la solución propuesta.

\section{Cliente}

\subsection{HTML}

Hypertext Markup Language o Lenguaje de etiquetado para hipertexto is un standard para la creación de paginas web y aplicaciones web. junto con CSS y JavaScript forman las bases tecnológicas de la web.

HTML describe la estructura de la página web semanticamente e inicialmente le da una pista al navegador de como lucirá el contenido.

\subsection{CSS}
Cascading Style Sheets o planillas de estilos en cascada son usadas para describir la presentación de un documento escrito usando un lenguaje de etiquetado, usualmente HTML. 

CSS fue creado para una clara separación del contenido con la presentación, modificando el aspecto de las páginas, como tamaños, formas y colores de las distintas etiquetas.

\subsubsection{Bootstrap}
Es un marco 

\subsection{Javascript}



\section{Servidor}

\subsection{PHP}

Es un lenguaje de programación multipropósito mayormente usado para la programación web como lenguaje para la ejecución de tareas en el servidor. Originalmente creado por Rasmus Lerdorf en 1994 y ahora es producido por el Equipo de producción de PHP. 

Es famoso por su amigable curva de aprendizaje, junto con la facilidad que ofrece para combinarse con el lenguaje de etiquetado HTML. Él código es procesado por un interprete de PHP implementado como un módulo en el servidor, que combina este resultado con el esqueleto de la página. Es altamente portable por ser distribuido como software libre bajo la licencia PHP y trabaja en todos los servidores web en casi todos los sistemas operativos.
