\chapter{Marco Tecnológico}
\thispagestyle{empty} % Quitar el número

A continuación, se presentan los aspectos tecnológicos relacionados con el desarrollo del proyecto de pasantía, se describen brevemente y se indica su uso como parte de la solución propuesta.

\section{Cliente}

\subsection{HTML}

Hypertext Markup Language o Lenguaje de etiquetado para hipertexto is un standard para la creación de paginas web y aplicaciones web. junto con CSS y JavaScript forman las bases tecnológicas de la web.

HTML describe la estructura de la página web semanticamente e inicialmente le da una pista al navegador de como lucirá el contenido.


\subsection{CSS}
Cascading Style Sheets o planillas de estilos en cascada son usadas para describir la presentación de un documento escrito usando un lenguaje de etiquetado, usualmente HTML. 

CSS fue creado para una clara separación del contenido con la presentación, modificando el aspecto de las páginas, como tamaños, formas y colores de las distintas etiquetas.

\subsection{Javascript}
Es un lenguaje de programación de alto nivel, debilmente tipado, multipropósito e interpretado. Es usado para darle interactividad a las páginas web, desde un sistema web hasta videojuegos. Es soportado por la mayoria de los navegadores actuales que implementan su propia representación de la especificación ECMAScript.

\subsection{Ajax}
Abreviación de Asynchronous Javascript and XML es un conjunto de técnicas que hacen uso de varias tecnologias en el lado del cliente para crear aplicaciones web asíncronas. Con Ajax, las aplicacoines web pueden mandar y hacer peticiones de datos sin interferir con la visualización y el comportamineto de la página existente cambiando el contenido de la página dinámicamente sin tener que recargarla enteramente.

\subsection{JQuery}
Es una libreria multiplataforma de Javascript diseñada para simplificar la escritura de programas que ejecuten uniformemente en los distintos navegadores. Es gratis amparado en la permisiva licencia del MIT. Además permite la creación de módulos encima de Javascript que pueden ser compartidos para las funciones más básicas y comunes en la programación web.


\subsection{Bootstrap}
Es un Framework de HTML, CSS y Javascript que facilita la construcción de páginas web que se adapten al ambiente donde son mostradas por medio de un sistema de cuadrícula con 12 columnas. Además provee algunos de los elemetos usuales en las interfaces modernas como acordiones, paginación, botones desplegables que facilitan la construcción de interfaces web multiplataforma en distintos navegadores.


\section{Servidor}

\subsection{PHP}

Es un lenguaje de programación multipropósito mayormente usado para la programación web como lenguaje para la ejecución de tareas en el servidor. Originalmente creado por Rasmus Lerdorf en 1994 y ahora es producido por el Equipo de producción de PHP. 

Es famoso por su amigable curva de aprendizaje, junto con la facilidad que ofrece para combinarse con el lenguaje de etiquetado HTML. Él código es procesado por un interprete de PHP implementado como un módulo en el servidor, que combina este resultado con el esqueleto de la página. Es altamente portable por ser distribuido como software libre bajo la licencia PHP y trabaja en todos los servidores web en casi todos los sistemas operativos.

\subsection{Smarty}
Smarty is a template engine for PHP. More specifically, it facilitates a manageable way to separate application logic and content from its presentation. This is best described in a situation where the application programmer and the template designer play different roles, or in most cases are not the same person.

\subsection{Microsoft SQL Server}

Es un sistema de administración de bases de datos relacionales desarrollado por Microsoft. Tiene la función primaria de guardar y servir datos  que sean requeridos por otras aplicaciones, que pueden correr en la misma computadora o en otra a traves de una conexión en red.

\subsection{Servidor HTTP Apache}

Es un servidor web gratis y de código abierto ofrecido bajo los terminos de la licencia Apache 2.0. Procesa las peticiones que llegan via HTTP, el protocolo básico de la web. Apache es desarrollado y mantenido  por una comunidad de desarrolladores pertenecientes a la Fundación de Software Apache. Es multiplataforma, funciona tanto en sistemas UNIX asi como en Windows.


