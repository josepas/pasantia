\chapter{Marco Tecnológico}
\thispagestyle{empty} % Quitar el número

A continuación, se presentan los aspectos tecnológicos relacionados con el desarrollo del proyecto de pasantía, se describen brevemente y se indica su uso como parte de la solución propuesta.

\section{Cliente}

\subsection{HTML}

Hypertext Markup Language o Lenguaje de etiquetado para hipertexto es un estándar para la creación de páginas web y aplicaciones web. junto con CSS y JavaScript forman las bases tecnológicas de la web.

HTML describe la estructura de la página web semánticamente e inicialmente le da una pista al navegador de como lucirá el contenido.


\subsection{CSS}
Cascading Style Sheets o planillas de estilos en cascada son usadas para describir la presentación de un documento escrito usando un lenguaje de etiquetado, usualmente HTML. 

CSS fue creado para una clara separación del contenido con la presentación, modificando el aspecto de las páginas, como tamaños, formas y colores de las distintas etiquetas.

\subsection{Javascript}
Es un lenguaje de programación de alto nivel, débilmente tipado, multipropósito e interpretado. Es usado para darle interactividad a las páginas web, desde un sistema web hasta videojuegos. Es soportado por la mayoría de los navegadores actuales que implementan su propia representación de la especificación ECMAScript.

\subsection{AJAX}
Abreviación de \emph{Asynchronous} Javascript and XML es un conjunto de técnicas que hacen uso de varias tecnologías en el lado del cliente para crear aplicaciones web asíncronas. Con AJAX, las aplicaciones web pueden hacer y recibir peticiones de datos sin interferir con la visualización y el comportamiento de la página existente, cambiando el contenido de la página dinámicamente sin tener que recargarla enteramente.

\subsection{JQuery}
Es una librería multiplataforma de Javascript diseñada para simplificar la escritura de programas que ejecuten uniformemente en los distintos navegadores. Es gratis amparado en la permisiva licencia del MIT. Además, permite la creación de módulos encima de Javascript que pueden ser compartidos para las funciones más básicas y comunes en la programación web.


\subsection{Bootstrap}
Es un \emph{framework} de HTML, CSS y Javascript que facilita la construcción de páginas web que se adapten al ambiente donde son mostradas por medio de un sistema de cuadrícula con 12 columnas. Además provee algunos de los elementos usuales en las interfaces modernas como acordeones, paginación, botones desplegables que facilitan la construcción de interfaces web multiplataforma en distintos navegadores.


\section{Servidor}

\subsection{PHP}

Es un lenguaje de programación multipropósito mayormente usado para la programación web como lenguaje para la ejecución de tareas en el servidor. Originalmente creado por Rasmus Lerdorf en 1994 y ahora es producido por el Equipo de producción de PHP. 

Es famoso por su amigable curva de aprendizaje, junto con la facilidad que ofrece para combinarse con el lenguaje de etiquetado HTML. Él código es procesado por un intérprete de PHP implementado como un módulo en el servidor, que combina este resultado con el esqueleto de la página. Es altamente portable por ser distribuido como software libre bajo la licencia PHP y trabaja en todos los servidores web en casi todos los sistemas operativos.

\subsection{Smarty}
Smarty es un motor de maquetado para PHP. Específicamente, facilita la separación de la lógica de la aplicación y contenido de la presentación. Su mejor uso es descrito en la interacción de dos equipos de programación en el que uno genera las plantillas y el otro programa la aplicación, pero facilita la organización en desarrollos individuales. Haciendo usos de métodos como la inclusión y la herencia.

\subsection{Microsoft SQL Server}

Es un sistema de administración de bases de datos relacionales desarrollado por Microsoft. Tiene la función primaria de guardar y servir datos que sean requeridos por otras aplicaciones, que pueden correr en la misma computadora o en otra a través de una conexión en red.

\subsection{Servidor HTTP Apache}

Es un servidor web gratis y de código abierto ofrecido bajo los términos de la licencia Apache 2.0. Procesa las peticiones que llegan vía HTTP, el protocolo básico de la web. Apache es desarrollado y mantenido por una comunidad de desarrolladores pertenecientes a la Fundación de Software Apache. Es multiplataforma, funciona tanto en sistemas UNIX así como en Windows.

\subsection{Swift mailer}
Swift mailer es una librería de PHP que permite a cualquier aplicación escrita en este lenguaje el envío de correo electrónicos, proveyendo una interfaz flexible a través del paradigma orientado a objetos.

\section{Pruebas}

\subsection{PHP Unit}

Es un \emph{framework} para la realización de pruebas de software basado en la arquitectura de creación de pruebas xUnit.




