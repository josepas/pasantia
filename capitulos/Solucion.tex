\chapter{Solución del problema}
\thispagestyle{empty} % Quitar el número

En este capítulo se expresan las técnicas utilizadas por el pasante para el cumplimiento de las funcionalidades expresadas en el capítulo anterior.

El lenguaje de programación usado como base para el Sistema de Gestión de Aprendizaje (SGA) de Fischer, Knochblauch \& CO (FKC) es PHP. La mayor parte del código escrito por el pasante para la solución es por lo tanto escrita en este lenguaje.

El trabajo se dividió en módulos. Se analizaron sus dependencias con el fin de crear un plan de trabajo organizado, que permitiera además la entrega constante de software funcional, apegandose al marco de trabajo iterativo.

AQUI SERIA BUENO UNA INTRODUCCION CON CUANTOS MODULOS SE VAN A DESCRIBIR LUEGO PERO LA METO AL FINAL PORQUE NO ESTOY MUY CLARO CUAL SERÍA UNA CLARA DIVISIÓN.

	\section{Ubicaciones} % (fold)
	\label{sec:ubicaciones}
	
	Para soportar esta funcionalidad en la base de datos se creó una tabla llamada ubicación con los datos que parecieron relevantes para el cliente, ilustrada en el anexo \ref{fig:baseDeDatosFinal}. Dicha ubicación estaría relacionada con las sesiones, pudiendo una ubicación alojar distintas sesiones. El SGA permite mantener distintos clientes para dar soporte de grandes compañías con filiales, por lo que las ubicaciones se construyen aisladas del grueso de la funcionalidad (los seminarios) para poder ser usadas luego en funcionalidades futuras o integrada en sistemas de los clientes activos de SGAs de FKC. Tienen una relación directa con la tabla cliente a través de una llave foránea. Para la clave primaria se usó un identificador creciente autogenerado.

	Se acordó que el manejo de las ubicaciones se llevara a cabo en el área de administración de la aplicación, por el usuario administrador. Se agrego una nueva entrada de manejo de ubicaciones en la interfaz del administrador con este fin.

	Se logró conseguir un producto gratis de esta aplicación que permitiera mostrar ubicaciones marcadas en un mapa generado por Google enganchado a un iframe en el sistema, con tan solo especificar medidas latitud y longitud, que tuvieron que ser agregadas luego a la tabla ubicación.

	Para integrar la visualización de mapas usando la aplicación Google maps de manera gratuita para la empresa, el usuario debía ingresar los valores de latitud y longitud de la ubicación deseada. Para esto el pasante ofreció como solución analizar gramaticalmente la URL mostrada en la aplicación web google maps por medio de expresiones regulares para extraer los valores necesarios, facilitando así al usuario el proceso de agregar una ubicación sin que la empresa tuviera que usar las funcionalidades pagas de google.

	% section ubicaciones (end)

	\section{Cursos del tipo Seminario} % (fold)
	\label{sec:cursos_del_tipo_seminario}
	
	Luego de examinar las tablas y entender el funcionamiento de los cursos y como son asignados a los grupos (Anexo \ref{fig:baseDeDatosPrevia}) se decidió extender esta tabla como una generalización.

	El sistema solo daba soporte a cursos del tipo multimedia dentro del mismo sistema, que podían ser de autoría del cliente o externos. Estos dos tipos eran manejados con un booleano ``cursoDeAutoria'' decisión de diseño que no daba espacio para la expansión de más tipos de cursos. El pasante sugirió agregar un campo ``tipo'' a la tabla de Curso asignándole el entero \emph{2} al tipo de curso seminario. dejando los enteros \emph{0} y \emph{1} a los anteriores tipos de cursos. La migración de los tipos anteriores para ser representados con el nuevo campo ``tipo'' fue sugerida pero los demás desarrolladores rechazaron la propuesta por provocar cambios en otras funcionalidades del sistema. el estado final de la base de datos se demuestra en el anexo \ref{fig:baseDeDatosFinal}.

	Una vez diferenciado el nuevo tipo curso de los demás en la aplicación se procedió a dar soporte a las actividades CRUD para los seminarios. para esto se reutilizó la sección del sistema usada para la creación de cursos. Agregando la opción \emph{seminario} a las opciones previas en la creación de un curso y cambiando la estructura de la forma HTML usando Javascript, así como los datos enviados mediante el método POST y su respectiva validación en el servidor para el soporte de los seminarios. La vista se incluyó en el anexo \ref{fig:creacionSeminario}.

	Una de las diferencias de los seminarios con los cursos multimedia es la forma en la que son aprobados, se decidió que podían existir dos tipos:

	\begin{itemize}
		\item Aprobar con tan solo asistir al curso.
		\item Aprobar asistiendo al curso y aprobando un examen presencial.
	\end{itemize}

	Para el soporte de estos dos tipos se agregó un campo ``con\_examen'' que permite la representación.

	El pasante aprovechó los métodos de eliminación, listado y asignación a grupos que el sistema ya poseía, ajustándolos para el funcionamiento con los seminarios. 

	Para lograr esto debió separar visualmente los distintos tipos de cursos en las vistas de listado, para esto agrego un icono que los diferenciara en la vista de cursos del administrador (anexo \ref{fig:listarCursos}). Además, agregó un filtro persistente en la sesión HTTP que permitiera al administrador la visualización de solo cursos del tipo seminario a lo largo de su conexión.

	En el método asignación se agregó luego un \emph{preview} con las sesiones disponibles para el seminario visibles en el apéndice \ref{fig:asignarSeminario} además del icono para los seminarios. 

	% section cursos_del_tipo_seminario (end)

	\section{Usuario instructor} % (fold)
	\label{sec:usuario_instructor}
	
	Para el manejo de la puntuación de las sesiones y el manejo de la asistencia se decidió crear un nuevo tipo de usuario en el sistema que puediera apoyar al administrador en estas tareas. el usuario administrador tiene acceso a todas las funciones de un instructor en todos los cursos del sistema, mientras que el usuario solo tiene acceso a las sesiones en la que él mismo es él instructor.

	Para esto se debió modificar el módulo de \emph{login} para agregar la autentificación del nuevo usuario, que primero confirma si el usuario tiene una jerarquía más alta que la de instructor (un usuario de más rango que contenga sus funciones), en caso contrario, verifica si el usuario que intenta ingresar es instructor de algún curso.

	El manejo de la autorización es realizado de forma manual, así que solo se debió poblar una estructura del tipo diccionario con las funcionalidades que se consideraran necesarias para el usuario instructor. En este caso, administrar los cursos de los que es instructor, funcionalidad desarrollada en el \emph{sprint} cinco.

	
	% section usuario_instructor (end)




