\chapter{Marco Teórico}
\thispagestyle{empty} % Quitar el número

En el presente capítulo es presentado y descrito el conjunto de conceptos, términos y pilares teóricos relevantes sobre el cuales se basó este proyecto.

\section{Conceptos básicos sobre el área de trabajo}

\subsection{E-learning o aprendizaje electrónico}

Se denomina aprendizaje electrónico (conocido también por el anglicismo \emph{e-learning}) a la educación a distancia completamente virtualizada a través de los nuevos canales electrónicos (las nuevas redes de comunicación, en especial Internet), utilizando para ello herramientas o aplicaciones de hipertexto (correo electrónico, páginas web, foros de discusión, mensajería instantánea, plataformas de formación, etc.) como soporte de los procesos de enseñanza-aprendizaje.

Gracias a las nuevas tecnologías de la información y la comunicación (TIC), los estudiantes ``en línea'' pueden comunicarse y colaborar con sus compañeros ``de clase'' e instructores (profesores, tutores, mentores, etc.), de forma síncrona o asíncrona, sin limitaciones espacio-temporales. Es decir, se puede entender como una modalidad de aprendizaje dentro de la educación a distancia en la que se utilizan las redes de datos como medios (Internet, intranets, etc.), las herramientas o aplicaciones hipertextuales como soporte (por ejemplo, correo electrónico, web, chat, etc. ) y los contenidos y/o unidades de aprendizaje en línea como materiales formativos. Como ejemplo, simples imágenes, audio, video, documentos, llegando hasta complejas producciones multimedia.

Las ventajas que ofrece la formación en línea son las siguientes:

Eliminación de barreras espaciales y temporales (desde su propia casa, en el trabajo, en un viaje a través de dispositivos móviles, etc.). Supone una gran ventaja para discentes geográficamente dispersos o alejados.

Prácticas en entornos de simulación virtual, difíciles de conseguir en formación presencial, sin una gran inversión.

Enriquecimiento colectivo del proceso de aprendizaje sin límites geográficos.
Actualización constante de los contenidos (deducción lógica del punto anterior).
Reducción de costos (en la mayoría de los casos, a nivel metodológico y, siempre, en el aspecto logístico).


Es una alternativa de formación que no reemplaza necesariamente a los profesores y las clases presenciales, sino que es un espacio que desarrolla la autonomía del aprendiz.

\subsection{B-learning o aprendizaje híbrido}

En un concepto más relacionado con lo semipresencial, se encuentra el llamado b-learning (blended learning). Esto significa que un curso dictado en este formato incluirá tanto clases presenciales como actividades de e-learning.

Este modelo de formación hace uso de las ventajas de la formación 100\% en línea y la formación presencial, combinándolas en un solo tipo de formación que agiliza la labor tanto del formador como del alumno. La enseñanza combinada o mezclada, a veces también denominada enseñanza híbrida se define como cursos o programas en los que el contenido online supone entre un 30\% y un 70\% del total del curso. En contraste, la enseñanza presencial incluye aquellos cursos en los que el contenido online oscila entre 0\% y 29\% del curso. Las ventajas que se suelen atribuir a esta modalidad de aprendizaje son la unión de las dos modalidades que combina:

las ya comentadas que se atribuyen al e-learning y las de la formación presencial como: aplicación de los conocimientos e interacción física, lo cual tiene una incidencia notable en la motivación de los participantes, facilita el establecimiento de vínculos, y ofrece la posibilidad de realizar actividades algo más complicadas de realizar de manera puramente virtual.

\subsection{Sistema de Gestión de Aprendizaje}

Un sistema de gestión de contenidos es un programa que permite crear una estructura de soporte para la creación y administración de contenidos por parte de los participantes principalmente en páginas web. El entorno de hardware y software diseñado para automatizar y gestionar el desarrollo de actividades formativas se conoce como plataforma de teleformación o sistema de gestión de aprendizaje.

Este tipo de plataformas tecnológicas también se conoce como LMS (Learning Management System). Un LMS registra usuarios, organiza catálogos de cursos, almacena datos de los usuarios y provee informes para la gestión. Suelen incluir también herramientas de comunicación al servicio de los participantes en los cursos. 

Actualmente existe una gran oferta de plataformas, tanto de comerciales como de código abierto. En el ámbito universitario se está implantando con gran aceptación la plataforma de licencia libre Moodle usada actualmente en la Universidad Simón Bolívar. En el ámbito comercial es muy famosa actualmente la plataforma Blackboard.

\section{Desarrollo de \emph{software}}

Desarrollo de software es el proceso de programación, documentación y prueba necesario para la creación y mantenimineto de aplicaciónes y marcos de desarrollo resultando en un producto de software. Es un proceso en donde se escribe y se mantiene una base de código fuente, pero en un sentido mas amplio, incluye todas las etapas desde la concepción hasta la manifistación final del producto, usualmente siendo un proceso planeado y estructurado enmarcado en una metodología. Esta área puede incluir, investigación, desarrollo, prototipado, modificación, reuso, reingeniería, mantenimiento, asi como otras actividades que resulten en un producto de software.

El software puede ser desarrollado con una variedad de propósitos, los tres mas comunes siendo clumplir necesidades de un cliente o empresa específica, para cubrir la necesidad de algun grupo de usuarios o para el uso personal.

\subsection{Modelo Vista Controlador}

Modelo-vista-controlador (MVC) is un patron de arquitectura de software para la implementación de interfaces de usuario. Divide la aplicación en tres partes interconectadas. Esto se hace con el fin de separar representaciones internas de la infromación con la forma en la que esta se presenta al usuario. Este patrón permite disociar los componentes promoviendo el reuso del código y el desarrollo en paralelo.

Tradicionalmente fue usado para la construcción de de aplicaciones gráficas de escritorio, pero se ha vuelto popular para el diseño de aplicaciones web e incluso móviles. Los lenguajes de progamación populares de la época poseen marcos de trabajo que facilitan la implatanción de aplicaciones web usando este patrón.

\begin{itemize}

\item El Modelo: Es la representación de la información con la cual el sistema opera, por lo tanto gestiona todos los accesos a dicha información, tanto consultas como actualizaciones, implementando también los privilegios de acceso que se hayan descrito en las especificaciones de la aplicación (lógica de negocio). Envía a la vista la parte de la información que en cada momento se le solicita para que sea mostrada.

\item La Vista: Presenta la información en un formato adecuado para que el usuario pueda interactuar con ella.

\item El Controlador: definido como la interfaz o intermediario entre la vista y el modelo. Su principal labor es la de responder a eventos e invocar peticiones al modelo cuando se hace alguna solicitud. 

\end{itemize}

\subsection{Arquitectura cliente-servidor}

La arquitectura cliente-servidor es un modelo de aplicación distribuida en el que las tareas se reparten entre los proveedores de recursos o servicios, llamados servidores, y los demandantes, llamados clientes. Un cliente realiza peticiones a otro programa, el servidor, quien le da respuesta.

En esta arquitectura la capacidad de proceso está repartida entre los clientes y los servidores, aunque son más importantes las ventajas de tipo organizativo debidas a la centralización de la gestión de la información y la separación de responsabilidades, lo que facilita y clarifica el diseño del sistema.

La separación entre cliente y servidor es una separación de tipo lógico, donde el servidor no se ejecuta necesariamente sobre una sola máquina ni es necesariamente un sólo programa. existen distintos tipos específicos de servidores incluyen los servidores web, los servidores de bases de datos, los servidores del correo, etc. Mientras que sus propósitos varían de unos servicios a otros, la arquitectura básica seguirá siendo la misma.

\subsection{Framework o entorno de trabajo} 

En el área de la computación un \emph{framework} es una abstracción de software que provee una funcionalidad generica que puede ser cambiada y adaptada con código escrito por el usuario para generar soluciones específicas. Provee facilidades y estándares para la creación y despliege de aplicaciones en distintos nichos de la programación. Pueden incluir programas de soporte, compiladores, librerías de código, herramientas y APIs que conjugan todos los componentes para permitir el desarrollo de un proyecto o sistema.


