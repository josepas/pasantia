\section{Sexto sprint} % (fold)
\label{sec:sexto_sprint}

\subsection{Objetivos}

\begin{itemize}
	\item Implementar las funcionalidades de calificación de las sesiones de seminarios.
	\item Generar \gls{PDF} con la lista de estudiantes de una sesión.
	\item Implementar funcionalidad para hacer el módulo de seminarios opcional.
\end{itemize}

En este sprint, una vez que se contó con confirmaciones válidas que procedían desde usuarios aprendices del sistema se procedió a implementar la calificación por parte del usuario instructor y administrador. 

\subsection{Actividades} % (fold)
\label{sub:actividades6}

\begin{itemize}

\item Calificación una sesión


\item Actualizar la funcionalidad de borrado de una sesión

Se decidió agregar condiciones para el borrado de las sesiones al desarrollar esta funcionalidad. Para borrar una sesión esta no debe tener usuarios que la hayan confirmado. Tampoco debe tener resultados de ningún usuario. Borrar una sesión con resultados eliminaría también el estado de aprobado que pudieran tener algunos usuarios.

\item Generación de PDF

Se permitió además al usuario instructor la opción de imprimir una lista con los estudiantes de la sesión que contiene los datos de la sesión y los estados posibles de cada uno dependiendo del tipo de seminario. El resultado se muestra en el Anexo \ref{fig:listarAlumnos}.


\item Módulo de seminarios como una opción

Dado que el SGA es un producto que se personaliza para los distintos clientes de la compañía, los clientes se interesaron en tener la opción de crear una nuevo SGA sin la funcionalidad del módulo seminario creado. Para poder ofrecerlo con una opción extra que pudiera flexibilizar el esquema de precios del producto.

Para esto el se modificó la funcionalidad de otro tipo de usuario: \emph{super admin}. Específicamente la creación y actualización de un cliente. Dependiendo de si los cursos del tipo seminario estaban activados o no, todas las opciones relacionadas con este módulo debían desaparecer. Entre estas:

\begin{itemize}
	\item Creación de un curso del tipo seminario.
	\item Manejo de las ubicaciones si éstas solo son usadas para sesiones de seminarios.
	\item Visualización de los cursos del tipo seminario y sus sesiones a lo largo de la aplicación.
	\item Desactivación del nuevo tipo de usuario \emph{instructor}.
\end{itemize}

\end{itemize}


% section sexto_sprint (end)
