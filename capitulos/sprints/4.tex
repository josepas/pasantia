\section{Cuarto sprint} % (fold)
\label{sec:cuarto_sprint}

\subsection{Objetivos}

\begin{itemize}
	\item Desarrollar el módulo de sesiones de seminario.
	\item Integrar el nuevo módulo con los demás módulos del sistema.
	\item Agregar visualización de sesiones activas para una ubicación.
\end{itemize}

En este sprint se desarrolló el módulo de gestión de las sesiones de un seminario representado en el diagrama de casos de uso (anexo \ref{fig:diagramaCasosDeUso}), la parte más importante del proyecto. Interactúa con los demás módulos implementados y es la funcionalidad central que da vida al sistema. Es la información más importante que se guarda en la base de datos, Los datos de las sesiones y la interacción de los usuarios con éstas.

Cabe destacar que al momento de mencionar sesión se hace referencia a las sesiones de los seminarios y no la sesión HTTP a menos que se especifique lo contrario.

\subsection{Actividades} % (fold)
\label{sub:actividades4}

\begin{itemize}


\item Representación en la base de datos

\item Creación del CRUD

El proceso básico de creación del CRUD fue muy parecido al CRUD de las ubicaciones, pero con más consideraciones a tener en cuenta, pues aquí se integraban distintas partes de la aplicación, como un instructor, que debía ser un usuario del sistema; una ubicación, en la que se llevaría a cabo la sesión creada, manejo de fechas, activación y desactivación de las sesiones. La vista de CRUD se presenta en el anexo \ref{fig:creacionSesion}.

\item Creación de una sesión

\item Actualización de una sesión

\item Borrado de una sesión

El borrado básico se realizó en este sprint, pero luego tuvo restricciones que surgieron del manejo de las calificaciones implementado en el sexto sprint y además se agregaron opciones de notificación a los usuarios de una sesión cancelada en el sprint siete. 

\item Visualización de sesiones activas de una ubicación

Se agregó para facilitar el proceso de elegir una locación libre en la fecha de la sesión.

\end{itemize}


% section cuarto_sprint (end)
