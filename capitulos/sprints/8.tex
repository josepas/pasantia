\section{Octavo sprint} % (fold)
\label{sec:octavo_sprint}

\subsection{Objetivos}

\begin{itemize}
	\item Integrar el módulo de seminarios al SGA de Bibliomed.
\end{itemize}

A mitad del desarrollo del módulo uno de los clientes de FKC (Bibliomed) poseedor de su SGA se interesó por el módulo de seminarios. Por lo que la primera tarea al finalizar el desarrollo fue integrar las funcionalidades al sistema personalizado para esta compañía.

Se decidió entonces hacer las modificaciones necesarias para el soporte de estas peticiones en la base de datos del SGA de Bibliomed. El resultado final quedó plasmado en el Anexo \ref{fig:baseDeDatosBibliomed}.

\subsection{Actividades} % (fold)
\label{sub:actividades8}

\begin{itemize}

\item Adaptación de la base de datos

\item Asignación de un seminario

\item Modificación de la interfaz del aprendiz

La interfaz del listado de los módulos disponibles para el aprendiz se adaptó para el soporte de los seminarios la interfaz anterior se muestra en el Anexo \ref{fig:bibliomedViejo} y la interfaz final en el Anexo \ref{fig:reservarSesionBiliomed}.

\item Aprobación de los módulos

El mayor desafío de este sprint fue la programación de la aprobación de los módulos. Para esto se trabajó muy de cerca con el desarrollador del SGA de Bibliomed, asegurándose que se modificaba correctamente el algoritmo para la aprobación de un módulo de aprendizaje.

Se proveyó una interfaz para la fácil consulta de terminación de un seminario dentro de un módulo y juntos unieron las culminaciones de los dos tipos de cursos para cumplir las restricciones pedidas por el cliente para la aprobación de los módulos.

\item Integración con el calendario del usuario

\end{itemize}


% section octavo_sprint (end)
