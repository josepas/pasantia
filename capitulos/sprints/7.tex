\section{Septimo sprint} % (fold)
\label{sec:septimo_sprint}

\subsection{Objetivos}

\begin{itemize}
	\item Crear nuevo tipo de sesiones: \emph{sesiones en línea}.
	\item Implementar notificaciones para los eventos del módulo.
	\item Integrar los cursos del tipo seminario a las estadísticas del sistema.
\end{itemize}

Al mitad del desarrollo del sprint anterior el cliente consideró agregar una nueva funcionalidad al módulo. La posibilidad de agregar sesiones dentro de los seminarios que pudieron darse en conferencias en linea. El pasante decidió incluir esta funcionalidad en el presente sprint que ya contenía el envio de notificaciones por correo en los principales eventos del sistema y la integración de los seminarios en las estadísticas de los cursos.

\subsection{Actividades} % (fold)
\label{sub:actividades7}

\subsubsection{Sesiones en línea en la base de datos}

Añadir sesiones en linea fue un proceso parecido al de agregar los seminarios a los cursos. Se especificaron los datos que diferenciaban a las sesiones en línea de las ya existentes y se decidió por lo tanto crear una especialización de la tabla Sesión en física y ``en linea'' (anexo \ref{fig:baseDeDatosFinal}).

La diferencia entre estas sesiones y las anteriores es que estas ya no tendrian asignadas una ubicación física, en este caso tendrian una URL y un codígo de acceso para ingresar a la misma. se ingresó un campo tipo a la Tabla Sesión para diferenciarlas. el entero cero representa las sesiones físicas y el uno se refiere a las en línea.

\subsubsection{Presentación de sesiones en línea }

La lógica de negocio detras de las sesiones en linea terminó siendo la misma que para las sesiones físicas por lo que la calificación de éstas no debió ser modificada. El reto estuvó en la creación, actualización y presentación de las mismas al usuario.

Para crear una sesión online se agregó una opción de tipo en la creación de las sesiones (anexo \ref{fig:creacionSesion}) al igual que con los cursos.

Este tipo de sesiones debian ser mostradas de forma distinta en las vistas que involucraran las sesiones ya que estas cuentan con datos distintos. por lo que se modificó la visualización del antiguo modal del aprendiz para la ubicación. En el caso de una sesión física se seguiría mostrando el mismo modal y en caso contrario se mostraria el modal con los datos referentes a la sesión en línea solo en el caso en que el usuario haya confirmado dicha sesión.

En lugares donde se listan las sesiones el campo de ubicación paso a llamarse ``ubicación/url'' y en el caso de una sesión en linea se muestra e texto ``sesión en línea'' en vez de el nombre de la ubicación. 

% section septimo_sprint (end)