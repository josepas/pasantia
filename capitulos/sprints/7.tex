\section{Séptimo sprint} % (fold)
\label{sec:septimo_sprint}

\subsection{Objetivos}

\begin{itemize}
	\item Crear nuevo tipo de sesiones: \emph{sesiones en línea}.
	\item Implementar notificaciones para los eventos del módulo.
	\item Integrar los cursos del tipo seminario a las estadísticas del sistema.
\end{itemize}

A la mitad del desarrollo del sprint anterior el cliente consideró agregar una nueva funcionalidad al módulo. La posibilidad de agregar sesiones dentro de los seminarios que pudieron darse en conferencias en línea. El pasante decidió incluir esta funcionalidad en el presente sprint que ya contenía el envío de notificaciones por correo en los principales eventos del sistema y la integración de los seminarios en las estadísticas de los cursos.

Con este sprint se finalizó el desarrollo del módulo de seminarios excluyendo futuras correcciones.

\subsection{Actividades} % (fold)
\label{sub:actividades7}

\begin{itemize}

\item Sesiones en línea en la base de datos

La diferencia entre estas sesiones y las anteriores es que estas ya no tendrían asignadas una ubicación física, en este caso tendrían una \gls{URL} y un código de acceso para ingresar a la misma. 

\item Presentación de las sesiones en línea

La lógica de negocio detrás de las sesiones en línea terminó siendo la misma que para las sesiones físicas por lo que la calificación de éstas no debió ser modificada. El reto estuvo en la creación, actualización y presentación de las mismas al usuario.

\item Notificaciones

Se procedió primero a identificar cuáles eran los eventos que requerían de notificaciones a los usuarios entre estos se encontraron:

\begin{itemize}
	\item Información de una sesión confirmada o cancelada por el usuario.
	\item Correo informando al usuario que una de sus sesiones confirmadas fue cancelada.
	\item Notificar al instructor que un usuario ha confirmado o cancelado una de sus sesiones.
\end{itemize}

\item Integración de las estadísticas

Las estadísticas es una de las partes más importantes en el manejo de los cursos (Anexo \ref{fig:estadisticasAdmin}). Allí los administradores pueden corroborar el desempeño de los aprendices. Al ser los seminarios un nuevo tipo de curso también debían ser incluidos en las estadísticas. La empresa proveyó un mapa de los estados de los cursos al que el pasante tuvo que moldear los resultados de las sesiones. 

\end{itemize}


% section septimo_sprint (end)
