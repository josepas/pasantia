\section{Quinto sprint} % (fold)
\label{sec:quinto_sprint}

\subsection{Objetivos}

\begin{itemize}
	\item Diseñar la interfaz para la gestión de sesiones.
	\item Desarrollar las funcionalidades para el aprendiz.
\end{itemize}

En este sprint luego de haber realizado los requerimientos básicos necesarios para el soporte del módulo o lo que generalmente es llamado \emph{backend} de la aplicación. Se procedió a realizar el \emph{frontend} que permitiera a los usuarios interactuar con los cursos del tipo seminario y sus sesiones. Específicamente los casos de uso pertenecientes al usuario aprendiz en la parte baja del diagrama de casos de uso (Anexo \ref{fig:diagramaCasosDeUso}).

\subsection{Actividades} % (fold)
\label{sub:actividades5}

\begin{itemize}

\item Diseño de la interfaz
\item Listado de las sesiones disponibles
\item Confirmación y cancelación de una sesión
\item Exportar sesión al calendario
\item Integración con el módulo de mensajería interna del sistema
\item Modal con los datos de la ubicación

	Se integró una ventana modal para mostrar directamente el mapa y los datos de la ubicación que eran previamente mostrados en el módulo de ubicación. Usando el mapa el usuario puede guardar la ubicación directamente en su cuenta \emph{google} si se encuentra autenticado con esta aplicación.

\end{itemize}


% section quinto_sprint (end)
