\section{Noveno sprint} % (fold)
\label{sec:noveno_sprint}

\subsection{Objetivos}

\begin{itemize}
	\item Migrar el SGA base de FKC para el manejo de caracteres UTF-8.
	\item Implantar el SGA de FKC en los ambientes Munich y producción.
	\item Implantar el SGA de Bibliomed en el ambiente de producción.
\end{itemize}

Hasta el momento todo el desarrollo se había realizado en servidores de prueba, en este sprint se traslada la base de código actualizada a los servidores de producción de cada empresa y se realizan las pruebas correspondientes.

\subsection{Actividades} % (fold)
\label{sub:actividades9}

\begin{itemize}

\item Migración para el soporte de caracteres UTF-8

Al integrar las nuevas funcionalidades en el SGA de Bibliomed el pasante se dio cuenta que muchos de los caracteres especiales de lenguaje alemán perdían representación. Esto se debió a que el SGA de Bibliomed representaba caracteres en el formato UTF-8 en comparación con el más viejo ISO del sistema base.

Bibliomed al soportar el lenguaje ruso usa la mejor opción UTF-8. Pero mantener este esquema significaría que los próximos sistemas vendidos por FKC también podrían luego necesitar la migración a UTF-8 para soportar otros lenguajes. El equipo decidió entonces asignar la tarea al pasante de migrar el sistema al soporte de caracteres UTF-8 en su versión básica.

Para lograr esto debían guardarse todos los archivos con el nuevo formato que previamente estaban guardados en el formato ISO y cambiar la etiqueta \gls{HTML} que especifica el \emph{charset} en todos los archivos del sistema. Se utilizaron expresiones regulares dentro del editor de texto para modificar este parámetro satisfactoriamente.

Los otros desarrolladores realizaron las pruebas pertinentes para comprobar el resultado de la operación.

\item Implantación del SGA de FKC

El proceso de implantación del sistema consistió en modificar las tablas en el servidor objetivo y enviar los archivos a través de \gls{FTP}. Este era un proceso totalmente manual.

Para FKC debió hacerse en dos servidores distintos, uno para la filial en Munich y el otro para producción.

Se realizaron pruebas manuales para la integración del módulo con las demás partes del sistema.

\item Implantación del SGA Bibliomed

El proceso fue análogo a la implantación del SGA de FKC solo cuidando las diferencias en la estructura de la base de datos.

\end{itemize}


% section noveno_sprint (end)
