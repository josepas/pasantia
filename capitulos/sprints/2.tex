\section{Segundo sprint} % (fold)
\label{sec:segundo_sprint}

\subsection{Objetivos}

\begin{itemize}
	\item Desarrollo del módulo ubicaciones que sirvan como locación de los seminarios.
\end{itemize}

En este sprint se desarrolló el módulo de manejo de ubicaciones representado en el diagrama de casos de uso (Anexo \ref{fig:diagramaCasosDeUso}). Se decidió iniciar con este módulo por ser una funcionalidad aislada, sencilla y componente necesario para la creación de sesiones presenciales. Tomando así un acercamiento de abajo hacia arriba en el desarrollo del proyecto.

\subsection{Actividades} % (fold)
\label{sub:actividades2}

\begin{itemize}

\item Ampliación de la base de datos

\item Creación del \gls{CRUD}


Se procedió entonces a la construcción de interfaces que permitieran la creación de una ubicación nueva, listar las ubicaciones existentes, modificar una ubicación existente tanto como eliminar una ubicación. Se hizo énfasis en que las interfaces creadas siguieran un aspecto consistente con las otras funciones de administración ilustradas en los apéndices \ref{fig:listarUbicaciones} y \ref{fig:editarUbicaciones}.

\item Integración con google maps

Al terminar la funcionalidad básica el dueño del producto sugirió integrar las ubicaciones del sistema con la aplicación Google maps. Tarea para la cual se investigó sobre el uso de este \gls{API}. 

En el Anexo \ref{fig:editarUbicaciones} se muestra como se integró el mapa a la vista de editar ubicación.

\end{itemize}





