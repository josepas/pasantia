\section{Segundo sprint} % (fold)
\label{sec:segundo_sprint}

\subsection{Objetivos}

\begin{itemize}
	\item Desarrollo del módulo ubicaciones que sirvan como locación de los seminarios.
\end{itemize}

En este sprint se desarrollo el módulo de manejo de ubicaciones representando en el diagrama de casos de uso (anexo \ref{fig:diagramaCasosDeUso}). Se decidió iniciar con este módulo por ser una funcionalidad aislada, sencilla y componente necesario para la creación de sesiones prescenciales. Tomando así un acercamiento de abajo hacia arriba en el desarrollo del proyecto.

\subsection{Actividades} % (fold)
\label{sub:actividades}

\subsubsection{Amplicación de la base de datos}

Para soportar esta funcionalidad en la base de datos se creo una tabla llamada ubicación con los datos que parecieron relevantes para el cliente ilutrada en el anexo \ref{fig:baseDeDatosFinal}. Dicha ubicación estaría relacionada con las sesiones, pudiendo una ubicación alojar distintas sesiones. El SGA permite mantener distintos clientes para dar soporte de grandes compañias con filiales, por lo que las ubicaciones se construyen aisladas del grueso de la funcionalidad (los seminarios) para poder ser usadas luego en funcionalidades futuras o integrada en sistemas de los clientes activos de SGAs de FKC. Tienen una relación directa con la tabla cliente a través de una llave foránea. Para la clave primaria se uso un identificador creciente autogenerado.

\subsubsection{Creación del CRUD}

Se acordó que el manejo de las ubicaciones se llevara acabo en el área de administración de la aplicación, por el usuario administrador. Se agrego una nueva entrada de manejo de ubicaciones en la interfaz del adminstrador con este fin.

Se procedió entonces a la construcción de interfaces que permitieran la creación de una ubicación nueva, listar las ubicaciones existentes, modificar una ubicación existente tanto como eliminar una ubicación. Se hizo enfasis en que las interfaces creadas siguieran un aspecto consistente con las otras funciones de administración.

\subsubsection{Integración con google maps}

Al terminar la funcionalidad básica el dueño del producto sugirió integrar las ubicaciones del sistema con la aplicación google maps. Tarea para la cual el pasante debió investigar sobre el uso de este API. Se logró conseguir un producto gratis de esta aplicación que pemitiera mostrar ubicaciones marcadas en un mapa generado por google enganchado a un iframe en el sistema, con tan solo especificar medidas latitud y longitud, que tuvieron que ser agregadas luego a la tabla ubicación.

Para mantener el uso de esta funcionalidad de manera gratuita para la empresa el usuario debia ingresar los valores de latitud y longitud de la ubicación deseada, para esto el pasante ofreció como solución analizar gramaticalmente la URL mostrada en la aplicación web google maps por medio de expresiones regulares para extraer los valores necesarios, facilitando así al usuario el proceso de agregar una ubicación sin que la empresa tuviera que usar las funcionalidades pagas de google.

Se recomienda en un futuro el pago de este API (Javascript de google maps) si se desea facilitar aun más la interación con mapas dentro de la aplicación. Con el beneficio de no depender que en algun momento el proveedor google modifique la estructura de sus URL o desaparezca los datos de latitud y longitud de las mismas.

\subsubsection{Pruebas de software}

Se realizaron pruebas de software para el ingreso de campos vacios, cantidades negativas, datos de tipos erroneos en los diferentes campos y campos no llenados.






