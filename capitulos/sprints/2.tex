\section{Segundo sprint} % (fold)
\label{sec:segundo_sprint}

\subsection{Objetivos}

\begin{itemize}
	\item Desarrollo del módulo ubicaciones que sirvan como locación de los seminarios.
\end{itemize}

En este sprint se desarrolló el módulo de manejo de ubicaciones representado en el diagrama de casos de uso (anexo \ref{fig:diagramaCasosDeUso}). Se decidió iniciar con este módulo por ser una funcionalidad aislada, sencilla y componente necesario para la creación de sesiones presenciales. Tomando así un acercamiento de abajo hacia arriba en el desarrollo del proyecto.

\subsection{Actividades} % (fold)
\label{sub:actividades2}

\subsubsection{Ampliación de la base de datos}

\subsubsection{Creación del CRUD}


Se procedió entonces a la construcción de interfaces que permitieran la creación de una ubicación nueva, listar las ubicaciones existentes, modificar una ubicación existente tanto como eliminar una ubicación. Se hizo énfasis en que las interfaces creadas siguieran un aspecto consistente con las otras funciones de administración ilustradas en los apéndices \ref{fig:listarUbicaciones} y \ref{fig:editarUbicaciones}.

\subsubsection{Integración con google maps}

Al terminar la funcionalidad básica el dueño del producto sugirió integrar las ubicaciones del sistema con la aplicación Google maps. Tarea para la cual el pasante debió investigar sobre el uso de este API. 

Se recomienda en un futuro el pago de este API (Javascript de google maps) si se desea facilitar aún más la interacción con mapas dentro de la aplicación. Con el beneficio de no depender que en algún momento el proveedor google modifique la estructura de sus URL o desaparezca los datos de latitud y longitud de las mismas.

En el anexo \ref{fig:editarUbicaciones} se muestra como se integró el mapa a la vista de editar ubicación.






