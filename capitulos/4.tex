\chapter{Marco Metodológico}
\thispagestyle{empty} % Quitar el número

En este capítulo se describe la metodología de trabajo utilizada en el desarrollo de este proyecto de pasantía.

Con el fin de enmarcar un desarrollo orientado a mejorar la productividad y calidad del software, que involucre además una reducción de riesgos y se adapte a las necesidades del cliente, se selecciona una metodología de desarrollo de software que moldea la construcción de características y funcionalidades a ofrecer por parte del software a través de prácticas de desarrollo que se adapten al mismo.

\section{Naturaleza del proyecto}
El trabajo realizado por el pasante fue de extensión de un software existente. Una versión base de un SGA (Sistema de Gestión de Aprendizaje) que luego es instanciada para el uso de los distintos clientes de la compañía. 

Las funcionalidades realizadas, en principio no tuvieron un cliente que especificara los requerimientos. Por lo tanto, los módulos se realizaron con la colaboración del equipo interno de la empresa que fungieron como dueños del producto y con la visión de agregar valor al sistema y equipararlo con otros sistemas del mercado.

Los requerimientos, al no estar fijados desde el inicio claramente, tenían la posibilidad de cambiar a lo largo del desarrollo. El pasante considero como una buena táctica, dividir el proyecto en pequeñas entregas funcionales para así obtener retroalimentación sobre la dirección que tomaba el proyecto.

\section{Metodología ágil}

Para la realización de este proyecto se escogió el método de desarrollo ágil, que describe un grupo de principios para el desarrollo de software enmarcado en un ambiente en el que los requerimientos y las soluciones evolucionan a través del trabajo colaborativo entre los integrantes del equipo. Promueve planear adaptativamente, entregas tempranas, mejoramiento continuo, así como rápida y flexible respuesta al cambio.

Esta decisión fue tomada para aprovechar la flexibilidad que provee esta metodología. Este fue el elemento considerado como de mayor importancia dada la naturaleza del proyecto y se implementó en una de sus formas más comunes actualmente, SCRUM. A pesar de que SCRUM está planteado para ser usado por equipos de desarrollo fue ajustado para el proyecto en el que el pasante trabajo en su mayoría individualmente.

Esta metodología tiene un proceso de desarrollo iterativo incremental basado en entregas parciales y regulares del producto final al cliente, lo cual la hace flexible y de rápida adaptación ante cualquier cambio en cada iteración o Sprint. Dicha metodología está definida por los elementos descritos en las secciones siguientes de este capítulo.

\section{\emph{Scrum}}

Scrum es un \emph{framework} para la gestión del desarrollo de software aplicando la metodología ágil. Está diseñado para equipos de tres hasta nueve desarrolladores que dividen su trabajo en ciclos bisemanales llamados \emph{sprints}, chequean su progreso constantemente y entregan software funcional al final de cada fase.

Cada miembro de un equipo SCRUM tiene especificado uno de los siguientes roles dentro del mismo:

\subsection{Dueño del Producto o Product owner}
Es aquel miembro del equipo que administra y define los requisitos del proyecto de desarrollo de software, así como sus objetivos, agregando y organizando estos requisitos de acuerdo a prioridades para “maximizar el valor del producto”. Asimismo, representa a todas las personas interesadas en los resultados del mismo.

Para este proyecto de pasantías, el Ing. Harald Mais gerente de ventas dentro de FKC asumió el papel de Product Owner.

\subsection{Equipo}
Equipo de profesionales autoorganizado, multidisciplinario y con un sistema jerárquico horizontal que desarrollan el proyecto. Preferiblemente, el equipo debe estar compuesto con un número suficientemente pequeño de miembros como para mantener las características de “trabajo ágil”, pero lo suficientemente grande como para cumplir a tiempo todas las tareas.
Este proyecto de pasantía fue realizado de manera estrictamente individual, por lo que el pasante asumió el papel de Equipo.

\subsection{Facilitador o Scrum master}
Es la persona encargada de liderar al equipo en miras de que todos los procesos internos se lleven de la mejor manera, cumpliendo con las reglas de SCRUM, a lo largo de todo el desarrollo del proyecto. Sirve de mediador entre el equipo de desarrollo y el dueño del producto, facilitando las reuniones y eliminando los impedimentos que puedan presentarse durante el desarrollo.
Para este proyecto de pasantías, el Ing. Christian Ament asumió el papel de Scrum Master.

\subsection{Stakeholders o Partes interesadas}
Son aquellas personas para quienes el proyecto producirá el beneficio esperado que justifica su producción, pues son las interesadas en la realización del proyecto de desarrollo. Su participación se limita a las revisiones de cada sprint.

Miembros del equipo de programación y ventas formaban parte de las revisiones al final de cada sprint.

\subsection{Eventos}
Los eventos son todas aquellas reuniones planificadas para el seguimiento del proyecto de desarrollo y pueden ser:

\subsection{Sprint}
Es aquel período, de un tiempo previamente fijado y constante para todo el proyecto, durante el cual el equipo trabaja para convertir un subconjunto de requerimientos en una nueva versión del software totalmente operativo. Los sprints para este proyecto de pasantía tuvieron una duración aproximada de 10 días, por lo que se realizaron 10 sprints.

\subsection{Sprint Planning}
El Sprint Planning es una reunión que se realiza antes del inicio de cada Sprint, donde el equipo de desarrollo determina la carga de trabajo que se compromete a completar en ese sprint, realizando la planificación del mismo. Las reuniones se dieron entre el pasante, el Scrum master y el dueño del producto.

\subsection{Daily Scrum}
Reunión diaria, de máximo quince minutos, en la que el equipo informa sobre el estado del proyecto. Cada miembro responde a las siguientes tres preguntas:

\begin{itemize}
	\item 	¿Qué hiciste ayer?
	\item 	¿Qué harás hoy?
	\item 	¿Has tenido algún impedimento para alcanzar tu objetivo?
\end{itemize}

Debido a que la pasantía se realizó de forma individual, esta reunión no se llevó a cabo, aunque fueron preguntas que el pasante se realizó constantemente a lo largo del proyecto.

\subsection{Sprint Review}
Reunión que debe realizarse al final de cada sprint en la que el equipo de desarrollo presenta el trabajo completado durante el mismo a los interesados.

A los efectos de este proyecto de pasantía, esta reunión no se realizó de manera formal.

\subsection{Sprint Retrospective}
Después de cada sprint se lleva a cabo una retrospectiva del mismo, en la cual todos los miembros del equipo dan su opinión acerca del sprint recién superado en miras de mejorar continuamente el proceso de desarrollo. Como el proyecto de pasantía se realizó individualmente, esta reunión no se llevó a cabo.

\subsection{Artefactos}
La metodología Scrum hace uso de una serie de documentos que permiten su correcto funcionamiento y la comunicación del equipo completo. Entre ellos tenemos:

\subsection{Product Backlog}
Es un documento de alto nivel para todo el proyecto que consiste en una pila dinámica de requisitos denominados historias, descritos en un lenguaje no técnico y priorizados por valor de negocio. Decimos que es dinámica, pues los requisitos y prioridades se revisan y ajustan durante el curso del proyecto. Aquí, el Product Owner lista las características, funcionalidades, mejoras y correcciones del producto.

\subsection{Sprint Backlog}
Es un documento detallado y administrado por el equipo de desarrollo donde se describen, con una lista dinámica, todas las tareas a realizar para llevar a cabo las historias de un sprint.

\section{Pruebas de software}

Las pruebas del software no fueron requeridas por la empresa, por lo que no se les dio la atención necesaria. Pero al pertenecer al proceso de desarrollo de cualquier software de calidad el pasante decidió realizar las que estuvieron al alcance de las limitaciones de tiempo. Entre estas estuvieron unitarias de cada módulo y de integración entre los módulos desarrollados por el pasante, fue virtualmente imposible la creación de pruebas de integración con otros módulos del sistema debido a limitaciones de tiempo y a la ausencia total de pruebas de los demás módulos pertenecientes al sistema.

