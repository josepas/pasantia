\chapter{Entorno Empresarial}
\thispagestyle{empty} % Quitar el número

En este capítulo se describe el entorno empresarial en el cual tuvo lugar el desarrollo del proyecto de pasantía, la empresa Fischer, Knoblauch \& Co (FKC) filial de Frankfurt.
\section{Fischer, Knoblauch \& Co.}

Es un proveedor de servicios multimedia especializado en el área de aprendizaje electrónico. Está presente en Frankfurt y Munich en Alemania asi como en Basel, Suiza. Fundada en 1996 por Guy Fischer y Thomas Knoblauch.

\begin{figure}[h]
	\begin{center}
		\includegraphics[width=0.5\textwidth]{logos/logoFKC.jpg}
		\caption{Logo de la empresa FKC.} \label{fig:logoFKC}
	\end{center}
\end{figure}


Proveen consultoria en la integración y ampliación del aprendizaje electrónico a compañias de diversos sectores en Alemania y Bélgica. Se encargan de sugerir la elección de tecnologías, concepción del plan de aprendizaje, didácticas y metódologia de la enseñanza, producción del contenido audiovisual, hasta la integración de la solución en el ambiente del cliente. 

Basicamente, si una compañia requiere enseñar una cierta habilidad a sus empleados contacta a un proveedor de servicios de aprendizaje electrónico, FKC los ayuda a integrar un plan aprendizaje a su empresa, que se ven materializados en entrenamientos basados en la web. 

FKC también posee un Sistema de Gestión de Aprendizaje, la pieza de software en la que el pasante trabajó, que es personalizable y permite la organización de estos entrenamientos creados por la empresa

Además FKC haciendo uso de su departamento gráfico y programadores también provee servicios de posicionamiento empresarial en la web, mediante la creación de páginas, logos y demás contenido multimedia que la compañia requiera. 

Sus programadores día a día se enfrentan con diversos retos informáticos en distintos lenguajes de programación. Estos pueden ser: migraciones de sistemas de bases de datos, internazionalización de sus aplicaciones que llegan a estar hasta en 10 lenguajes distintos, diseño de soluciones multiplataforma y el manejo e instalación de Frameworks que faciliten la construcción de soluciones multimedia.

\section{Estructura organizacional}

En la figura \ref{fig:estructuraFKC} se muestra la estructura organizacional de FKC Frankfurt:

\begin{figure}[h]
\begin{center}
	\includegraphics[width=\textwidth]{figuras/estructuraFKC.jpg}
	\caption{Estructura organizacional de FKC.} \label{fig:estructuraFKC}
\end{center}
\end{figure}

FKC Frankfurt es un equipo multidiciplinario donde es importante la comunicación entre los distintos \emph{stakeholders}, los creadores de concepto, se unen a los diseñadores y los programadores para plasmar fielmente los requerimientos del cliente para obtener como resultado una solución hecha a la medida.

\section{Cargo ocupado por el pasante} 

El pasante perteneció al grupo de programación que se muestra en la figura \ref{fig:estructuraFKC} donde formó parte de un equipo de 6 programadores.



