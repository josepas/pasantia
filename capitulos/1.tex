\chapter{Entorno Empresarial}
\thispagestyle{empty} % Quitar el número

En este capítulo se describe el entorno empresarial en el cual tuvo lugar el desarrollo del proyecto de pasantía, la empresa \gls{FKC} filial de Frankfurt.

\section{Fischer, Knoblauch \& Co.}

Es un proveedor de servicios multimedia especializado en el área de aprendizaje electrónico. Está presente en Frankfurt y Munich en Alemania así como en Basel, Suiza. Fundada en 1996 por Guy Fischer y Thomas Knoblauch.

Proveen consultoría en la integración y ampliación del aprendizaje electrónico a compañías de diversos sectores en Alemania y Bélgica. Se encargan de sugerir la elección de tecnologías, concepción del plan de aprendizaje, didácticas y metodología de la enseñanza, producción del contenido audiovisual, hasta la integración de la solución en el ambiente del cliente. 

Si una compañía requiere enseñar una cierta habilidad a sus empleados contacta a un proveedor de servicios de aprendizaje electrónico, \gls{FKC} los ayuda a integrar un plan aprendizaje a su empresa, que se ven materializados en entrenamientos basados en la web. 

\gls{FKC} también posee un Sistema de Gestión de Aprendizaje, la pieza de \emph{software} en la que se trabajó, que es personalizable y permite la organización de estos entrenamientos creados por la empresa.

Además, \gls{FKC} haciendo uso de su departamento gráfico y programadores, provee servicios de posicionamiento empresarial en la web, mediante la creación de páginas, logos y demás contenido multimedia que la compañía requiera. 

Sus programadores día a día se enfrentan con diversos retos informáticos en distintos lenguajes de programación. Estos pueden ser: migraciones de sistemas de bases de datos, internacionalización de sus aplicaciones que llegan a estar hasta en diez lenguajes distintos, diseño de soluciones multiplataforma y el manejo e instalación de \emph{frameworks} que faciliten la construcción de soluciones multimedia.

\section{Estructura organizacional}

En la figura \ref{fig:estructuraFKC} se muestra la estructura organizacional de \gls{FKC} Frankfurt:

\begin{figure}[h]
\begin{center}
	\includegraphics[width=\textwidth]{figuras/estructuraFKC.jpg}
	\caption{Estructura organizacional de \gls{FKC}.} \label{fig:estructuraFKC}
\end{center}
\end{figure}

item

\gls{FKC} Frankfurt es un equipo multidisciplinario donde es importante la comunicación entre los distintos \emph{stakeholders}. Los creadores de concepto se unen a los diseñadores y los programadores para plasmar fielmente los requerimientos del cliente y obtener como resultado una solución hecha a la medida.

A continuación una descripción breve de los cargos en el organigrama en la figura \ref{fig:estructuraFKC}.

\begin{itemize}
	\item Director Ejecutivo es la persona de máxima autoridad, encargada de la gestión y dirección administrativa en la organización.

 	\item Director de Operaciones es el responsable del control de las actividades diarias de la corporación y de manejo de las operaciones. Reporta directamente al director ejecutivo.

	\item Secretaría encargada de dar apoyo a los empleados de la empresa en cuanto a la gestión de papeleo y la comunicación de las actividades.

	\item La cuadrilla de programación se encarga de materializar las peticiones que llegan de los demás departamentos mediante soluciones informáticas. El pasante perteneció a este grupo de 6 desarrolladores.

	\item El grupo de diseño gráfico se encarga de generar el material multimedia junto con los integrantes del grupo de concepción y los clientes. 

	\item El departamento de ventas se encarga del \emph{marketing} de los productos ofrecidos por \gls{FKC}.


\end{itemize}







