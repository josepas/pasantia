\chapter{Desarrollo de las funcionalidades}
\thispagestyle{empty} % Quitar el número

En este capítulo se describe el orden en el que las tareas fueron realizadas durante el desarrollo del proyecto de pasantía. Bajo las directrices de la metodología \emph{scrum} y a lo largo de diez sprints, comprendiendo las fases: especificación y análisis de requerimientos, diseño e implementación, e implantación de los cambios realizados a \gls{SGA} de \gls{FKC}. A continuación, se describen las actividades realizadas en cada fase, así como actividades fuera de la cobertura del plan de trabajo que se realizaron para \gls{FKC}.

\section{Primer sprint} % (fold)
\label{sec:primer_sprint}

\subsection{Objetivos}

\begin{itemize}
	\item Conocer el ambiente de trabajo de la empresa.
	\item Aprender a usar el lenguaje de programación PHP y sus buenas prácticas.
	\item Analizar a fondo el funcionamiento del SGA a extender.
	\item Levantar los requerimientos del proyecto a realizar.
\end{itemize}

\subsection{Actividades} % (fold)
\label{sub:actividades1}

\begin{itemize}
\item Familiarización con las herramientas

No se poseía de experiencia previa con el lenguaje de programación usado en la empresa, PHP, por lo que se acordó la exploración de referencias sobre el funcionamiento y el correcto uso de dicho lenguaje.

Se usaron distintos recursos tanto literarios como web, mayormente la página web que contiene la documentación oficial del lenguaje como referencia. \cite{bib:php}

\item Análisis a fondo el funcionamiento del SGA

Para esto se instalaron las herramientas comunes de desarrollo en inglés, puesto que se recibió un ambiente completamente en alemán. Entre estos: sistema operativo, manejador de las distintas bases de datos Microsoft SQL y Microsoft Access, y el navegador.

Una vez instalado el ambiente de desarrollo adecuado se procedió a explorar el sistema. Rápidamente se notó que el código fuente escrito estaba muy desorganizado. Código alto acoplamiento en el que se mezclaban lógica del negocio con la presentación. constante uso de instrucciones \gls{SQL} construidas dentro de cada vista susceptibles a inyecciones de \gls{SQL}. Muy bajo reúso de código a lo largo de la aplicación y técnicas de programación desactualizadas para el código PHP escrito en la actualidad especialmente al momento de recuperar información de la base de datos. El código fuente no describía ninguno de los patrones de diseño que podían ayudar para la construcción de sistemas de este tipo, como composición, observador, entre otros. No existía para el sistema en cuestión ningún tipo de pruebas, ni documentación que apoyara esta exploración.

Se descubrió el uso del lenguaje de maquetado Smarty que permite la separación de la capa lógica y la de presentación y se procedió a conseguir referencias para el aprendizaje de esta librería.

Se estudió además el esquema de la base de datos usando la herramienta \emph{SQL Management Studio} que genera automáticamente un esquema visual de la base datos, donde se buscó entender los patrones con los que fue construida con el fin de mantener consistencia en las nuevas funcionalidades a desarrollar. Entre estas, implementación de las relaciones entre tablas, nombramiento de los campos, así como el tipo y tamaño de los mismos.

Asimismo, se analizó la estructura de los archivos, para mantener la misma estructura con la que estaban ordenados, separando los distintos componentes de la aplicación como archivos de código PHP, Javascript, \gls{CSS} y archivos estáticos. Se evidenció una estructura en el nombramiento de los archivos que se siguió a lo largo del desarrollo, colocando primero el nombre de lo que podría llamarse módulo y luego la acción específica dentro del mismo, por ejemplo: \emph{seminar\_session\_create, seminar\_session\_update, location\_create}, etc.

\item Levantamiento de requerimientos

Al terminar el análisis de la base de código y entender a grandes rasgos su funcionamiento y estructura se procedió a hacer el levantamiento de los requerimientos necesarios para la extensión. El objetivo era dividir el proyecto en piezas de funcionalidad con el fin de obtener una visión más clara y objetiva de las necesidades del cliente, así como un mapa que permitiera crear un plan y una estimación para la realización del proyecto. De esta reunión surgió el diagrama de casos de uso (anexo \ref{fig:diagramaCasosDeUso}.)

\item Exploración de otras plataformas

En esta fase también se realizó una investigación sobre la implementación de esta funcionalidad en otros SGA como e-front y moodle con el fin de tener una referencia de un producto que ya se encuentra en el mercado.

\end{itemize}






% subsection actividades (end)

\section{Segundo sprint} % (fold)
\label{sec:segundo_sprint}

\subsection{Objetivos}

\begin{itemize}
	\item 
\end{itemize}

\subsection{Actividades} % (fold)
\label{sub:actividades}


\section{Tercer sprint} % (fold)
\label{sec:tercer_sprint}

\subsection{Objetivos}

\begin{itemize}
	\item Analizar la estructura de cursos previamente implementados en el sistema.
	\item Desarrollar e integrar del módulo cursos del tipo seminario.
	\item Crear un nuevo usuario en el sistema para gestionar los cursos presenciales.
\end{itemize}

\subsection{Actividades} % (fold)
\label{sub:actividades3}

\begin{itemize}

\item Análisis de los cursos implementados en el sistema
\item Desarrollo del módulo cursos del tipo seminario
\item Creación del usuario instructor

\end{itemize}



% section tercer_sprint (end)

\section{Cuarto sprint} % (fold)
\label{sec:cuarto_sprint}

\subsection{Objetivos}

\begin{itemize}
	\item Desarrollar el módulo de sesiones de seminario.
	\item Integrar el nuevo módulo con los demás módulos del sistema.
	\item Agregar visualización de sesiones activas para una ubicación.
\end{itemize}

En este sprint se desarrolló el módulo de gestión de las sesiones de un seminario representado en el diagrama de casos de uso (anexo \ref{fig:diagramaCasosDeUso}), la parte más importante del proyecto. Interactúa con los demás módulos implementados y es la funcionalidad central que da vida al sistema. Es la información más importante que se guarda en la base de datos, Los datos de las sesiones y la interacción de los usuarios con éstas.

Cabe destacar que al momento de mencionar sesión se hace referencia a las sesiones de los seminarios y no la sesión \gls{HTTP} a menos que se especifique lo contrario.

\subsection{Actividades} % (fold)
\label{sub:actividades4}

\begin{itemize}


\item Representación en la base de datos

\item Creación del \gls{CRUD}

El proceso básico de creación del \gls{CRUD} fue muy parecido al \gls{CRUD} de las ubicaciones, pero con más consideraciones a tener en cuenta, pues aquí se integraban distintas partes de la aplicación, como un instructor, que debía ser un usuario del sistema; una ubicación, en la que se llevaría a cabo la sesión creada, manejo de fechas, activación y desactivación de las sesiones. La vista de \gls{CRUD} se presenta en el anexo \ref{fig:creacionSesion}.

\item Creación de una sesión

\item Actualización de una sesión

\item Borrado de una sesión

El borrado básico se realizó en este sprint, pero luego tuvo restricciones que surgieron del manejo de las calificaciones implementado en el sexto sprint y además se agregaron opciones de notificación a los usuarios de una sesión cancelada en el sprint siete. 

\item Visualización de sesiones activas de una ubicación

Se agregó para facilitar el proceso de elegir una locación libre en la fecha de la sesión.

\end{itemize}


% section cuarto_sprint (end)

\section{Quinto sprint} % (fold)
\label{sec:quinto_sprint}

\subsection{Objetivos}

\begin{itemize}
	\item Diseño de la interfaz para la gestión de sesiones.
	\item Desarrollar las funcionalidades para el aprendiz.
\end{itemize}

En este sprint luego de haber realizado los requerimientos básicos necesarios para el soporte del módulo o lo que generalmente es llamado \emph{backend} de la aplicación. Se procedió a realizar el \emph{frontend} que permitiera a los usuarios interactuar con los cursos del tipo seminario y sus sesiones. Específicamente los casos de uso pertenecientes al usuario aprendiz en la parte baja del diagrama de casos de uso (Anexo \ref{fig:diagramaCasosDeUso}).

\subsection{Actividades} % (fold)
\label{sub:actividades5}

\begin{itemize}

\item Diseño de la interfaz
\item Listado de las sesiones disponibles
\item Confirmación y cancelación de una sesión
\item Exportar sesión al calendario
\item Integración con el módulo de mensajería interna del sistema
\item Modal con los datos de la ubicación

	Se integró una ventana modal para mostrar directamente el mapa y los datos de la ubicación que eran previamente mostrados en el módulo de ubicación. Usando el mapa el usuario puede guardar la ubicación directamente en su cuenta \emph{google} si se encuentra autenticado con esta aplicación.

\end{itemize}


% section quinto_sprint (end)

\section{Sexto sprint} % (fold)
\label{sec:sexto_sprint}

\subsection{Objetivos}

\begin{itemize}
	\item Implentar las funcionalidades de calificación de las sesiones de seminarios.
	\item Generar PDF con la lista de estudiantes de una sesión.
	\item Implementar funcionalidad para hacer el módulo de seminarios opcional.
\end{itemize}

En este sprint, una vez que se contó con confirmaciones válidas que procedian desde usuarios aprendices del sistema se procedió a implementar la calificación por parte del usuario instructor y adiministrador. El estado de un usuario con respecto a una sesión se modela en la tabla auxiliar Sesión-Usuario que puede apreciarse en el anexo \ref{fig:baseDeDatosFinal}.

\subsection{Actividades} % (fold)
\label{sub:actividades6}

\subsubsection{Mantener la integridad de la base de datos}

Un usuario solo debe poseer una única entrada en la tabla Sesión-Usuario que lo relacione inequivocamente con una sesión, para esto se se utilizó como clave primaria de la tabla la combinación de las foráneas \emph{sesion\_id} y \emph{usuario\_id}. Además se implementó una excepción para arrojar un error al usuario y notificar al administrador del sistema por medio de una entrada en la bitácora del sistema si este caso llegara a suceder.

\subsubsection{Calificar una sesión}

Una vez asegurada la unicidad de la relación usuario cursa una sesión se puede proceder a calificarla. Tanto el administrador como el instructor pueden calificar una sesión. La calificación de la sesión permite al usuario aprendiz cambiar el estado del seminario que se encuentra en la tabla Curso. Para facilitar la explicación se construyo un diagrama de estados que demuestra los posibles cambios (anexo \ref{fig:diagramaEstadosSesion}). 

Para la calificación se construyeron dos \emph{endpoints} uno para la modificación de la asistencia y otro que soporta la modificación del estado aprobado/reprobado. que solo pueden ser alcanzados si la fecha de inicio de la sesión ha sido alcanzada.

El proceso puede resumirse en:

\begin{enumerate}
	% 1
	\item Un administrador asigna un seminario a un grupo compuesto por una cantidad de usuarios (anexo \ref{fig:asignarSeminario}). 
	
	% 2
	\item Una vez asignado el seminario, todas las sesiones que lo componen son asignadas como posibles para los usuarios del grupo a través de la interfaz de listar cursos del anexo \ref{fig:aprendizListarCursos}.

	% 3 
	\item El usuario puede confirmar cualquier sesión que no este llena.

	% 4 
	\item Con una sesión confirmada el usuario puede cancelarla y volver al punto 1 para elegir una sesión distinta.

	% 5
	\item El usuario que confirmó una sesión asiste a ella. El instructor por lo tanto puede marcar la asistencia en la interfaz contruida para ello que se muestra en el anexo \ref{fig:gestionarSesion}. Si el seminario no posee examen final, la sesión y por lo tanto el seminario, son aprobados.

	% 6
	\item Si el seminario tiene un examen presencial se le presenta la opción de notificar que el aprendiz aprobo el examen, en caso de dejar este campo en blanco se intuye que el usuario reprobó dicho examen.

	% 7
	\item En el caso de que el aprendiz sea reprobado el seminario se marca como reprobado hata que el usuario confirme otra sesión.

\end{enumerate}

\subsubsection{Actualizar la funcionalidad de borrado de una sesión}

Se decidió agregar condiciones para el borrado de las sesión al desarrollar esta funcionalidad. Para borrar una sesión esta no debe tener usuarios que la hayan confirmado. Tampoco debe tener resultados de ningun usuario. Borrar una sesión con resultados eliminaría también el estado de aprobado que pudieran tener algunos usuarios.

\subsubsection{Generación de PDF}

Se permitió además al usuario instructor la opción de imprimir una lista con los estudiantes de la sesión que contiene los datos de la sesión y los estados posibles de cada uno dependiendo del tipo de seminario. Esto se logró creando una planilla de estilos CSS distinta para la vista de impresión de la página. El resultado se muestra en el anexo \ref{fig:listarAlumnos}.


\subsubsection{Módulo de seminarios como una opción}

Dado que el SGA es un producto que se personaliza para los distintos clientes de la compañía
, los clientes se interesaron en tener la opción de crear una nuevo SGA sin la funcionalidad del módulo seminario creado por el pasante. Para poder ofrecerlo con una opción extra que pudiera flexibilizar el esquema de precios del producto.

Para esto el pasante debió modificar la funcionalidad de otro tipo de usuario: \emph{super admin}. Especificamente la creación y actualización de un cliente. Dependiendo de si los cursos del tipo seminario estaban activados o no, todas las opciones relacionadas con este módulo debian desapacerer. Entre estas:

\begin{itemize}
	\item Creación de un curso del tipo seminario.
	\item Manejo de las ubicaciones si éstas solo son usadas para sesiones de seminarios.
	\item Visualización de los cursos del tipo seminario y sus sesiones a lo largo de la aplicación.
	\item Desactivación del nuevo tipo de usuario creado por el pasante \emph{instructor}.
\end{itemize}


% section sexto_sprint (end)
\section{Septimo sprint} % (fold)
\label{sec:septimo_sprint}

\subsection{Objetivos}

\begin{itemize}
	\item Implentar las funcionalidades de calificación de las sesiones de seminarios.
	\item Generar PDF con la lista de estudiantes de una sesión.
	\item Implementar funcionalidad para hacer el módulo de seminarios opcional.
\end{itemize}

En este sprint, una vez que se contó con confirmaciones válidas que procedian desde usuarios aprendices del sistema se procedió a implementar la calificación por parte del usuario instructor y adiministrador. El estado de un usuario con respecto a una sesión se modela en la tabla auxiliar Sesión-Usuario que puede apreciarse en el anexo \ref{fig:baseDeDatosFinal}.

\subsection{Actividades} % (fold)
\label{sub:actividades6}

\subsubsection{Mantener la integridad de la base de datos}


% section septimo_sprint (end)
\section{Octavo sprint} % (fold)
\label{sec:octavo_sprint}

\subsection{Objetivos}

\begin{itemize}
	\item Integrar el módulo de seminarios al SGA de Bibliomed.
\end{itemize}

A mitad del desarrollo del módulo uno de los clientes de FKC (Bibliomed) poseedor de su SGA se interesó por el módulo de seminarios. Por lo que la primera tarea al finalizar el desarrollo fue integrar las funcionalidades al sistema personalizado para esta compañía.

Se decidió entonces hacer las modificaciones necesarias para el soporte de estas peticiones en la base de datos del SGA de Bibliomed. El resultado final quedó plasmado en el anexo \ref{fig:baseDeDatosBibliomed}.

\subsection{Actividades} % (fold)
\label{sub:actividades8}

\begin{itemize}

\item Adaptación de la base de datos

\item Asignación de un seminario

\item Modificación de la interfaz del aprendiz

La interfaz del listado de los módulos disponibles para el aprendiz se adaptó para el soporte de los seminarios la interfaz anterior se muestra en el anexo \ref{fig:bibliomedViejo} y la interfaz final en el anexo \ref{fig:reservarSesionBiliomed}.

\item Aprobación de los módulos

El mayor desafío de este sprint fue la programación de la aprobación de los módulos. Para esto el pasante trabajo muy de cerca con el desarrollador del SGA de Bibliomed, asegurandose que se modificaba correctamente el algoritmo para la aprobación de un módulo de aprendizaje.

El pasante proveyó una interfaz para la fácil consulta de terminación de un seminario dentro de un módulo y juntos unieron las culminaciones de los dos tipos de cursos para cumplir las restricciones pedidas por el cliente para la aprobación de los módulos.

\item Integración con el calendario del usuario

\end{itemize}


% section octavo_sprint (end)

\section{Noveno sprint} % (fold)
\label{sec:noveno_sprint}

Hello father soy el nueveno
% section noveno_sprint (end)
\section{Décimo sprint} % (fold)
\label{sec:decimo_sprint}

Soy yo papa el decimon

% section decimo_sprint (end)

\section{Actividades extra} % (fold)
\label{sec:actividades_extra}

Durante la realización del proyecto dada la naturaleza de la empresa como consultora informática de aprendizaje electrónico el pasante fue requerido para la realización de tareas extra de los diversos clientes de \gls{FKC}.

	\subsection{Exportar preguntas de un SGA} % (fold)
	\label{sub:exportar_preguntas_de_un_sga}

	% subsection exportar_preguntas_de_un_sga (end)

	Una funcionalidad para un sistema realizado con \gls{ASP}. Consintió en la implementación de una migración para los datos de uno de los clientes. Específicamente, las preguntas presentes en su base de datos. El pasante debió automatizar la generación del nuevo formato para las preguntas, marcando las distintas preguntas con sus opciones con un carácter especial marcando la respuesta correcta y separando las preguntas según su tipo (selección simple, compuesta). Luego sirviendo el resultado a través de la web en el sistema del cliente en un archivo que pudiera ser interpretado en las \emph{suites} ofimáticas comunes.

	Para la realización de esta tarea el pasante realizó la lógica enteramente dentro del lenguaje \gls{SQL}, Limitando el uso de \gls{ASP} al servicio de los resultados a través de la página web.

	% section actividades_extra (end)

	\subsection{Migración a UTF-8} % (fold)
	\label{sub:migracion_a_utf_8}
	
	La empresa al estar satisfecha con la migración de su \gls{SGA} base a la codificación \gls{UTF-8} requirió que el pasante migrara un \gls{SGA} de un cliente a esta codificación para el soporte del lenguaje turco.
	% subsection migracion_a_utf_8 (end)

	\subsection{Funcionalidad en modulo \emph{responsive}} % (fold)
	\label{sub:funcionalidad_en_modulo_responsive}
	
	El pasante apoyó a otro desarrollador en uno de sus proyectos, una herramienta para generar cursos o presentaciones dentro de uno de los \gls{SGA} de un cliente.

	La herramienta permite la construcción de diapositivas con distintos elementos de interfaz básicos, como, imágenes con texto, figuras, etc. El pasante codificó los parámetros de entrada y la presentación de imágenes con leyendas al pie o encabezando las mismas, enmarcado dentro del diseño \emph{resposive}.

	% subsection funcionalidad_en_modulo_responsive (end)

	\subsection{Imágenes de las vistas de login} % (fold)
	\label{sub:imagenes_de_las_vistas_de_login}
	
	Los \gls{SGA} vendidos por \gls{FKC} son personalizables, entre uno de los parámetros que pueden modificarse para la adaptación a los clientes son las imágenes de la vista de inicio de sesión, y el \emph{banner} que se muestra al tope de cada vista del sistema tanto en la vista de usuario como en la vista de administrador.

	Anteriormente, los desarrolladores dentro de \gls{FKC} recibían las imágenes necesarias para estas vistas y las adaptaban al sistema de manera manual y colocándolas en un fichero creado para este propósito en el servidor.

	El pasante agregó una funcionalidad para agregar estas imágenes y su color de fondo directamente desde la interfaz de super admin dentro del sistema que soporta la previsualización de la imagen escogida. Ofreciendo una manera más rápida y sencilla para la elección de dicha imagen, así como sus dimensiones y además permitiendo que los mismos clientes puedan modificar sus imágenes de inicio de sesión y \emph{banner}. 

	% subsection imagenes_de_las_vistas_de_login (end)

\section{Dificultades encontradas} % (fold)
\label{sec:dificultades_encontradas}

A lo largo de la realización del proyecto de pasantía se encontraron las siguientes dificultades:

\begin{itemize}
	\item El idioma fue una limitante para el pasante a la hora de hacer el análisis del sistema y del código a modificar en algunos casos. El código muchas veces no explicaba por si solo la funcionalidad que implementaba. Teniendo el pasante que acudir a los otros desarrolladores para la explicación de casos borde no apreciables en el análisis semántico. Además, comentarios y nombres de rutinas se encontraban escritos en alemán, idioma que el pasante no dominaba al nivel necesario para su completo entendimiento.

	\item El manejo de la internacionalización del sistema como fechas, husos horarios y lenguajes fue un reto interesante que el pasante debió sortear, siendo los clientes de diversos países dentro de la Unión Europea. 
	
	\begin{itemize}
		\item Para el manejo de fechas y husos horarios se aprovechó la interacción de los campos tipo \emph{datetime} de SQL Server y la clase \emph{DateTime} de PHP, que permitió configurar las locales dependiendo de las características del servidor donde se encuentre alojado el sistema.

		\item Para el manejo del texto en diferentes lenguajes el pasante manejó un base de datos de la empresa donde se encuentran todos los textos de la aplicación. Insertando y consultando los textos adecuados en cada ocasión. Los textos de la base de datos luego pasaban a ser parte de una función filtro de todos los textos del sistema. Dicha función detectaba el idioma guardado en la sesión \gls{HTTP} y mostraba el texto oportuno. El pasante ingresó todos los textos en inglés que luego fueron traducidos por colegas de la empresa.
	\end{itemize}

\end{itemize}



% section dificultades_encontradas (end)

