\chapter{Desarrollo de las funcionalidades}
\thispagestyle{empty} % Quitar el número

En este capítulo se describe el proceso de desarrollo del proyecto de pasantía. Realizado bajo las directrices de la metodología SCRUM y a lo largo de diez sprints, comprendiendo las fases: especificación y análisis de requerimientos, diseño e implementación, e implantación de los cambios realizados a Sistema de Gestión de Aprendizaje (SGA) de Fischer Knoblauch \& CO (FKC). A continuación, se describen las actividades realizadas en cada fase, las dificultades encontradas, artefactos generados y las soluciones tomadas a lo largo del desarrollo de cada sprint.

\section{Primer Sprint} % (fold)
\label{sec:primer_sprint}

\subsection{Objetivos}

\begin{itemize}
	\item Familiarizarse con el ambiente de trabajo de la empresa.
	\item Aprender a usar el lenguaje de programación PHP y sus buenas prácticas.
	\item Analizar a fondo el funcionamiento del SGA a extender.
	\item Levantamiento de requerimientos del proyecto a realizar.
\end{itemize}

\subsection{Actividades} % (fold)
\label{sub:actividades1}

\begin{itemize}
\item Familiarización con las herramientas

El pasante no poseía experiencia previa con el lenguaje de programación usado en la empresa, PHP, por lo que se acordó la exploración de referencias sobre el funcionamiento y el correcto uso de dicho lenguaje.

Se usaron distintos recursos tanto literarios como web, mayormente la página web que contiene la documentación oficial del lenguaje como referencia.

\item Análisis a fondo el funcionamiento del SGA

Para esto el pasante tuvo que instalar las herramientas comunes de desarrollo en inglés, puesto que recibió un ambiente completamente en alemán. Entre estos: sistema operativo, manejador de las distintas bases de datos Microsoft SQL y Microsoft Access, y el navegador.

Una vez instalado el ambiente de desarrollo adecuado el pasante procedió a explorar el sistema. Rápidamente se dio cuenta que el código fuente escrito estaba muy dglsesorganizado. Código alto acoplamiento en el que se mezclaban lógica del negocio con la presentación. constante uso de instrucciones \gls{SQL} construidas dentro de cada vista susceptibles a inyecciones de \gls{SQL}. Muy bajo reúso de código a lo largo de la aplicación y técnicas de programación desactualizadas para el código PHP escrito en la actualidad especialmente al momento de recuperar información de la base de datos. El código fuente no describía ninguno de los patrones de diseño que podían ayudar para la construcción de sistemas de este tipo, como composición, observador, entre otros. No existía para el sistema en cuestión ningún tipo de pruebas, ni documentación que apoyara al pasante en esta exploración.

Se descubrió el uso del lenguaje de maquetado Smarty que permite la separación de la capa lógica y la de presentación y se procedió a conseguir referencias para el aprendizaje de esta librería.

Se estudió además el esquema de la base de datos usando la herramienta \emph{SQL Management Studio} que genera automáticamente un esquema visual de la base datos, donde se buscó entender los patrones con los que fue construida con el fin de mantener consistencia en las nuevas funcionalidades a desarrollar. Entre estas, implementación de las relaciones entre tablas, nombramiento de los campos, así como el tipo y tamaño de los mismos.

Asimismo, se analizó la estructura de los archivos, para mantener la misma estructura con la que estaban ordenados, separando los distintos componentes de la aplicación como archivos de código PHP, Javascript, \gls{CSS} y archivos estáticos. Se evidenció una estructura en el nombramiento de los archivos que se siguió a lo largo del desarrollo, colocando primero el nombre de lo que podría llamarse módulo y luego la acción específica dentro del mismo, por ejemplo: seminar\_session\_create, seminar\_session\_update, location\_create, etc.

\item Levantamiento de requerimientos

Al terminar el análisis de la base de código y entender a grandes rasgos su funcionamiento y estructura se procedió a hacer el levantamiento de los requerimientos necesarios para la extensión. El objetivo era dividir el proyecto en piezas de funcionalidad con el fin de obtener una visión más clara y objetiva de las necesidades del cliente, así como un mapa que permitiera al pasante crear un plan y una estimación para la realización del proyecto. De esta reunión surgió el diagrama de casos de uso (anexo \ref{fig:diagramaCasosDeUso}.)

\item Exploración de otras plataformas

En esta fase también se realizó una investigación sobre la implementación de esta funcionalidad en otros SGA como e-front y moodle con el fin de tener una referencia de un producto que ya se encuentra en el mercado.

\end{itemize}






% subsection actividades (end)

\section{Segundo sprint} % (fold)
\label{sec:segundo_sprint}

\subsection{Objetivos}

\begin{itemize}
	\item Desarrollo del módulo ubicaciones que sirvan como locación de los seminarios.
\end{itemize}

En este sprint se desarrolló el módulo de manejo de ubicaciones representado en el diagrama de casos de uso (anexo \ref{fig:diagramaCasosDeUso}). Se decidió iniciar con este módulo por ser una funcionalidad aislada, sencilla y componente necesario para la creación de sesiones presenciales. Tomando así un acercamiento de abajo hacia arriba en el desarrollo del proyecto.

\subsection{Actividades} % (fold)
\label{sub:actividades2}

\subsubsection{Ampliación de la base de datos}

Para soportar esta funcionalidad en la base de datos se creó una tabla llamada ubicación con los datos que parecieron relevantes para el cliente, ilustrada en el anexo \ref{fig:baseDeDatosFinal}. Dicha ubicación estaría relacionada con las sesiones, pudiendo una ubicación alojar distintas sesiones. El SGA permite mantener distintos clientes para dar soporte de grandes compañías
s con filiales, por lo que las ubicaciones se construyen aisladas del grueso de la funcionalidad (los seminarios) para poder ser usadas luego en funcionalidades futuras o integrada en sistemas de los clientes activos de SGAs de FKC. Tienen una relación directa con la tabla cliente a través de una llave foránea. Para la clave primaria se usó un identificador creciente autogenerado.

\subsubsection{Creación del CRUD}

Se acordó que el manejo de las ubicaciones se llevara a cabo en el área de administración de la aplicación, por el usuario administrador. Se agrego una nueva entrada de manejo de ubicaciones en la interfaz del administrador con este fin.

Se procedió entonces a la construcción de interfaces que permitieran la creación de una ubicación nueva, listar las ubicaciones existentes, modificar una ubicación existente tanto como eliminar una ubicación. Se hizo énfasis en que las interfaces creadas siguieran un aspecto consistente con las otras funciones de administración ilustradas en los apéndices \ref{fig:listarUbicaciones} y \ref{fig:editarUbicaciones}.

\subsubsection{Integración con google maps}

Al terminar la funcionalidad básica el dueño del producto sugirió integrar las ubicaciones del sistema con la aplicación google maps. Tarea para la cual el pasante debió investigar sobre el uso de este API. Se logró conseguir un producto gratis de esta aplicación que permitiera mostrar ubicaciones marcadas en un mapa generado por google enganchado a un iframe en el sistema, con tan solo especificar medidas latitud y longitud, que tuvieron que ser agregadas luego a la tabla ubicación.

Para mantener el uso de esta funcionalidad de manera gratuita para la empresa el usuario debía ingresar los valores de latitud y longitud de la ubicación deseada, para esto el pasante ofreció como solución analizar gramaticalmente la URL mostrada en la aplicación web google maps por medio de expresiones regulares para extraer los valores necesarios, facilitando así al usuario el proceso de agregar una ubicación sin que la empresa tuviera que usar las funcionalidades pagas de google.

Se recomienda en un futuro el pago de este API (Javascript de google maps) si se desea facilitar aún más la interacción con mapas dentro de la aplicación. Con el beneficio de no depender que en algún momento el proveedor google modifique la estructura de sus URL o desaparezca los datos de latitud y longitud de las mismas.

En el anexo \ref{fig:editarUbicaciones} se muestra como se integró el mapa a la vista de editar ubicación.







\section{Tercer sprint} % (fold)
\label{sec:tercer_sprint}

\subsection{Objetivos}

\begin{itemize}
	\item Analizar la estructura de cursos previamente implementados en el sistema.
	\item Desarrollo e integración del módulo cursos del tipo seminario.
	\item Crear un nuevo usuario para el sistema gestionar solo cursos presenciales.
\end{itemize}

\subsection{Actividades} % (fold)
\label{sub:actividades3}

\begin{itemize}

\item Análisis de los cursos implementados en el sistema
\item Desarrollo del módulo cursos del tipo seminario
\item Creación del usuario instructor

\end{itemize}



% section tercer_sprint (end)

\section{Cuarto sprint} % (fold)
\label{sec:cuarto_sprint}

\subsection{Objetivos}

\begin{itemize}
	\item Desarrollar el módulo de sesiones de seminario.
	\item Integrar el nuevo módulo con los demás módulos del sistema.
	\item Agregar visualización de sesiones activas para una ubicación.
\end{itemize}

En este sprint se desarrolló el módulo de gestión de las sesiones de un seminario, representado en el diagrama de casos de uso (Anexo \ref{fig:diagramaCasosDeUso}). Ésta es la parte más importante del proyecto, donde se describen las sesiones y su interacción con los usuarios.

Cabe destacar que al momento de mencionar sesión se hace referencia a las sesiones de los seminarios y no la sesión \gls{HTTP} a menos que se especifique lo contrario.

\subsection{Actividades} % (fold)
\label{sub:actividades4}

\begin{itemize}


\item Representación en la base de datos

\item Creación del \gls{CRUD}

El proceso básico de creación del \gls{CRUD} fue muy parecido al \gls{CRUD} de las ubicaciones, pero con más consideraciones a tener en cuenta, pues aquí se integraban distintas partes de la aplicación, como un instructor, que debía ser un usuario del sistema; una ubicación, en la que se llevaría a cabo la sesión creada, manejo de fechas, activación y desactivación de las sesiones. La vista de \gls{CRUD} se presenta en el Anexo \ref{fig:creacionSesion}.

\item Creación de una sesión

\item Actualización de una sesión

\item Borrado de una sesión

El borrado básico se realizó en este sprint, luego tuvo restricciones que surgieron del manejo de las calificaciones implementado en el sexto sprint. Además, se agregaron opciones de notificación a los usuarios de una sesión cancelada en el sprint siete. 

\item Visualización de sesiones activas de una ubicación

Se agregó para facilitar el proceso de elegir una locación libre en la fecha de la sesión.

\end{itemize}


% section cuarto_sprint (end)

\section{Quinto sprint} % (fold)
\label{sec:quinto_sprint}

5to sprint

% section quinto_sprint (end)
\section{Sexto sprint} % (fold)
\label{sec:sexto_sprint}

\subsection{Objetivos}

\begin{itemize}
	\item Implentar las funcionalidades de calificación de las sesiones de seminarios.
	\item Generar PDF con la lista de estudiantes de una sesión.
	\item Implementar funcionalidad para hacer el módulo de seminarios opcional.
\end{itemize}

En este sprint, una vez que se contó con confirmaciones válidas que procedian desde usuarios aprendices del sistema se procedió a implementar la calificación por parte del usuario instructor y adiministrador. El estado de un usuario con respecto a una sesión se modela en la tabla auxiliar Sesión-Usuario que puede apreciarse en el anexo \ref{fig:baseDeDatosFinal}.

\subsection{Actividades} % (fold)
\label{sub:actividades6}

\subsubsection{Mantener la integridad de la base de datos}

Un usuario solo debe poseer una única entrada en la tabla Sesión-Usuario que lo relacione inequivocamente con una sesión, para esto se se utilizó como clave primaria de la tabla la combinación de las foráneas \emph{sesion\_id} y \emph{usuario\_id}. Además se implementó una excepción para arrojar un error al usuario y notificar al administrador del sistema por medio de una entrada en la bitácora del sistema si este caso llegara a suceder.

\subsubsection{Calificar una sesión}

Una vez asegurada la unicidad de la relación usuario cursa una sesión se puede proceder a calificarla. Tanto el administrador como el instructor pueden calificar una sesión. La calificación de la sesión permite al usuario aprendiz cambiar el estado del seminario que se encuentra en la tabla Curso. Para facilitar la explicación se construyo un diagrama de estados que demuestra los posibles cambios (anexo \ref{fig:diagramaEstadosSesion}). 

Para la calificación se construyeron dos \emph{endpoints} uno para la modificación de la asistencia y otro que soporta la modificación del estado aprobado/reprobado. que solo pueden ser alcanzados si la fecha de inicio de la sesión ha sido alcanzada.

El proceso puede resumirse en:

\begin{enumerate}
	% 1
	\item Un administrador asigna un seminario a un grupo compuesto por una cantidad de usuarios (anexo \ref{fig:asignarSeminario}). 
	
	% 2
	\item Una vez asignado el seminario, todas las sesiones que lo componen son asignadas como posibles para los usuarios del grupo a través de la interfaz de listar cursos del anexo \ref{fig:aprendizListarCursos}.

	% 3 
	\item El usuario puede confirmar cualquier sesión que no este llena.

	% 4 
	\item Con una sesión confirmada el usuario puede cancelarla y volver al punto 1 para elegir una sesión distinta.

	% 5
	\item El usuario que confirmó una sesión asiste a ella. El instructor por lo tanto puede marcar la asistencia en la interfaz contruida para ello que se muestra en el anexo \ref{fig:gestionarSesion}. Si el seminario no posee examen final, la sesión y por lo tanto el seminario, son aprobados.

	% 6
	\item Si el seminario tiene un examen presencial se le presenta la opción de notificar que el aprendiz aprobo el examen, en caso de dejar este campo en blanco se intuye que el usuario reprobó dicho examen.

	% 7
	\item En el caso de que el aprendiz sea reprobado el seminario se marca como reprobado hata que el usuario confirme otra sesión.

\end{enumerate}

\subsubsection{Actualizar la funcionalidad de borrado de una sesión}

Se decidió agregar condiciones para el borrado de las sesión al desarrollar esta funcionalidad. Para borrar una sesión esta no debe tener usuarios que la hayan confirmado. Tampoco debe tener resultados de ningun usuario. Borrar una sesión con resultados eliminaría también el estado de aprobado que pudieran tener algunos usuarios.

\subsubsection{Generación de PDF}

Se permitió además al usuario instructor la opción de imprimir una lista con los estudiantes de la sesión que contiene los datos de la sesión y los estados posibles de cada uno dependiendo del tipo de seminario. Esto se logró creando una planilla de estilos CSS distinta para la vista de impresión de la página. El resultado se muestra en el anexo \ref{fig:listarAlumnos}.


\subsubsection{Módulo de seminarios como una opción}

Dado que el SGA es un producto que se personaliza para los distintos clientes de la compañia, los clientes se interesaron en tener la opción de crear una nuevo SGA sin la funcionalidad del módulo seminario creado por el pasante. Para poder ofrecerlo con una opción extra que pudiera flexibilizar el esquema de precios del producto.

Para esto el pasante debió modificar la funcionalidad de otro tipo de usuario: \emph{super admin}. Especificamente la creación y actualización de un cliente. Dependiendo de si los cursos del tipo seminario estaban activados o no, todas las opciones relacionadas con este módulo debian desapacerer. Entre estas:

\begin{itemize}
	\item Creación de un curso del tipo seminario.
	\item Manejo de las ubicaciones si éstas solo son usadas para sesiones de seminarios.
	\item Visualización de los cursos del tipo seminario y sus sesiones a lo largo de la aplicación.
	\item Desactivación del nuevo tipo de usuario creado por el pasante \emph{instructor}.
\end{itemize}


% section sexto_sprint (end)
\section{Septimo sprint} % (fold)
\label{sec:septimo_sprint}

\subsection{Objetivos}

\begin{itemize}
	\item Crear nuevo tipo de sesiones: \emph{sesiones en línea}.
	\item Implementar notificaciones para los eventos del módulo.
	\item Integrar los cursos del tipo seminario a las estadísticas del sistema.
\end{itemize}

Al mitad del desarrollo del sprint anterior el cliente consideró agregar una nueva funcionalidad al módulo. La posibilidad de agregar sesiones dentro de los seminarios que pudieron darse en conferencias en linea. El pasante decidió incluir esta funcionalidad en el presente sprint que ya contenía el envio de notificaciones por correo en los principales eventos del sistema y la integración de los seminarios en las estadísticas de los cursos.

Con este sprint se finalizó el desarollo del módulo de seminarios excluyendo futuras correcciones.

\subsection{Actividades} % (fold)
\label{sub:actividades7}

\subsubsection{Sesiones en línea en la base de datos}

Añadir sesiones en linea fue un proceso parecido al de agregar los seminarios a los cursos. Se especificaron los datos que diferenciaban a las sesiones en línea de las ya existentes y se decidió por lo tanto crear una especialización de la tabla Sesión en física y ``en linea'' (anexo \ref{fig:baseDeDatosFinal}).

La diferencia entre estas sesiones y las anteriores es que estas ya no tendrian asignadas una ubicación física, en este caso tendrian una URL y un codígo de acceso para ingresar a la misma. se ingresó un campo tipo a la Tabla Sesión para diferenciarlas. el entero cero representa las sesiones físicas y el uno se refiere a las en línea.

\subsubsection{Presentación de las sesiones en línea }

La lógica de negocio detras de las sesiones en linea terminó siendo la misma que para las sesiones físicas por lo que la calificación de éstas no debió ser modificada. El reto estuvó en la creación, actualización y presentación de las mismas al usuario.

Para crear una sesión online se agregó una opción de tipo en la creación de las sesiones (anexo \ref{fig:creacionSesion}) al igual que con los cursos.

Este tipo de sesiones debian ser mostradas de forma distinta en las vistas que involucraran las sesiones ya que estas cuentan con datos distintos. por lo que se modificó la visualización del antiguo modal del aprendiz para la ubicación. En el caso de una sesión física se seguiría mostrando el mismo modal y en caso contrario se mostraria el modal con los datos referentes a la sesión en línea solo en el caso en que el usuario haya confirmado dicha sesión.

En lugares donde se listan las sesiones el campo de ubicación paso a llamarse ``ubicación/url'' y en el caso de una sesión en linea se muestra e texto ``sesión en línea'' en vez de el nombre de la ubicación. 

\subsubsection{Notificaciones}

Se procedió primero a identificar cuales eran los eventos que requerían de notificaciones a los usuarios entre estos se encontraron:

\begin{itemize}
	\item Información de una sesión confirmada o cancelada por el usuario.
	\item Correo informando al usuario que una de sus sesiones confirmadas fue cancelada.
	\item Noficar al instructor que un usuario a confirmado o cancelado una de sus sesiones.
\end{itemize}

Las notificaciones en el sistema generalmente son hechas a través de correos, el pasante decidió mantener ésta práctica. Se utilizó para este propósito la libreria \emph{swift mailer}.

El proceso de agregar las notificaciones paso por crear una función de envoltura a la librerias de \emph{swift mailer} que recibiera los datos básicos necesarios para enviar un correo como: encabezado, mensaje y destinatarios dentro de un arreglo.

El resto fue ubicar los envios de correos en los sitios adecuados, llenando los datos correctos según fuera el caso.

Al comenzar a probar el sistema, los clientes se dieron cuenta que las notificaciones podian llegar a ser demasiadas en cursos compuestos por muchos estudiantes, por lo que luego se desarrollo una opción para desactivar los correos recibidos por el instructor de una sesión. como ejemplo se muestra uno de los correos enviados por la aplicación en el anexo \ref{fig:correos}.

\subsubsection{Integración de las estadísticas}

Las estadísticas es una de las partes mas importantes en el manejo de los cursos (anexo \ref{fig:estadisticasAdmin}). Allí los administradores pueden corroborar el desempeño de los aprendices. Al ser los seminarios un nuevo tipo de curso tambien debian ser incluidos en las estadísticas. La empresa proveyó un mapa de los estados de los cursos al que el pasanté tuvo que moldear los resultados de las sesiones. 

Debió modificar algunas consultas en las estadísticas para que estas puedieran reflejar los resultados de los seminarios. generalmente suavizando condiciones de los \emph{joins}. Se diferenciaron los seminarios con el uso de un icono distintivo.

Además, algunos datos en la tabla de visualización de las estadísticas no corcodaban con los datos referentes a los seminarios. Estos datos se omitieron como: momento de inicio, momento de terminación, tiempo de aprendizaje que mas adecuados y facilmente medibles en los cursos web.


% section septimo_sprint (end)
\section{Octavo sprint} % (fold)
\label{sec:octavo_sprint}

\subsection{Objetivos}

\begin{itemize}
	\item Crear nuevo tipo de sesiones: \emph{sesiones en línea}.
\end{itemize}

A mitad del desarrollo del módulo uno de los clientes de FKC (Bibliomed) poseedor de su SGA se interesó por el módulo de seminarios. Por lo que la primera tarea al finalizar el desarrollo fue integrar las funcionalidades al sistema personalizado para esta compañia.

En este sprint cuando se habla de cliente se hace referencia a la empresa Bibliomed.

El SGA de Bibliomed poseía una abstración por encima de los cursos llamada Módulos, que hizo que la integración no fuera tan trivial como era esperado. Además, el cliente agrego nuevas condiciones a la interacción de los seminarios.

Los Módulos no son más que agrupaciones de cursos en línea que deben ser realizados para su finalización. El cliente especificó que los seminarios podian ser un reemplazo o un corequisito a los cursos en línea, es decir, según los requerimientos del creador del módulo las formas de aprobarlo serian:

\begin{enumerate}
	\item Aprobar todos los cursos en línea.
	\item Aprobar el seminario asociado al Módulo.
	\item Aprobar ambos, cursos en línea y el seminario.
\end{enumerate}

Se decidió entonces hacer las modificaciones necesarias para el soporte de estas peticiones en la base de datos del SGA de Bibliomed. El resultado final quedo plasmado en el anexo \ref{fig:baseDeDatosBibliomed}.

\subsection{Actividades} % (fold)
\label{sub:actividades8}

\subsubsection{Adaptación de la base de datos}

La relación entre módulos y seminarios es 1 a 1, es decir, solo puede existir un seminario dentro de cada módulo, de igual manera el esquema se realizo con la flexibilidad necesaria para permitir más seminarios por módulo en el futuro. Para esto se colocó la foránea identificador del módulo en la tabla Seminario y no en los dos como es comunmente implementado. Que un seminario no pueda se asignado a dos módulos distintos se asegura a través de la interfaz. Un cambio a una interfaz más permisiva puede hacer que se necesite especificar la foránea de los dos lados de la relación. 

Se agrego también el campo ``modo\_aprobacion'' para representar los modos de aprobación descritos anteriormente.

El resto del esquema termina siendo el mismo que en el SGA base de FKC. uno de los objetivos de la integración fue alejarse lo menos posible del sistema base para facilitar la mantenibilidad.

\subsubsection{Asignación de un seminario}

En el SGA de Bibliomed la creación de un curso del tipo seminario paso a ser la asignación de un seminario a un módulo. Como los seminarios solo pueden existir asignados a un módulo, la asignación se hace desde la vista de edición del módulo como se demuestra en el anexo \ref{fig:creacionSeminarioBibliomed} en constraste con la creación de curso del tipo seminario del anexo \ref{fig:creacionSeminario}.

Al realizarse desde la vista de actualización del módulo se asegura que el seminario exista solo en ese módulo, ya que no puede elegirse un seminario previamente creado.

\subsubsection{Modificación de la interfaz del aprendiz}

La interfaz del listado de los módulos disponibles para el aprendiz se adaptó para el soporte de los seminarios la interfaz anterior se muestra en el anexo \ref{fig:bibliomedViejo} y la interfaz final en el anexo \ref{fig:reservarSesionBiliomed}.

\subsubsection{Aprobación de los módulos}

El mayor desafio de este sprint fue la programación de la aprobación de los módulos. Para esto el pasante trabajo muy de cerca con el desarrollador del SGA de Bibliomed, asegurandose que se modificaba correctamente el algoritmo para la aprobación de un módulo de aprendizaje.

El pasante proveyó una interfaz para la fácil consulta de terminación de un seminario dentro de un módulo y juntos unieron las culminaciones de los dos tipos de cursos para cumplir las restricciones pedidas por el cliente para la aprobación de los módulos.

\subsubsection{Integración con el calendario del usuario}

El pasante debió además hacer una integración de las sesiones confirmadas por el usuario con un módulo de calendario para el usuario implementado en el SGA de Bibliomed. Agregó a las consultas del previas que llenaban el calendario una consula que ofreciera las consultas confirmadas y atendidas por el usuario de manera que se mostraran en su calendario personal. Cuidando mantener la misma estructura de los eventos para su correcta presentación.

% section octavo_sprint (end)
\section{Noveno sprint} % (fold)
\label{sec:noveno_sprint}

\subsection{Objetivos}

\begin{itemize}
	\item Migrar el SGA base de FKC para el manejo de caracteres UTF-8.
	\item Implantar el SGA de FKC en los ambientes Munich y producción.
	\item Implantar el SGA de Bibliomed en el ambiente de producción.
\end{itemize}

Hasta el momento todo el desarrollo se había realizado en servidores de prueba, en este sprint se traslada la base de código actualizada a los servidores de producción de cada empresa y se realizan las pruebas correspondientes.

\subsection{Actividades} % (fold)
\label{sub:actividades9}

\begin{itemize}

\item Migración para el soporte de caracteres UTF-8

Al integrar las nuevas funcionalidades en el SGA de Bibliomed muchos de los caracteres especiales del lenguaje alemán perdían representación. Esto se debió a que el SGA de Bibliomed representaba caracteres en el formato UTF-8 en comparación con el más viejo ISO del sistema base.

Bibliomed al soportar el lenguaje ruso usa la mejor opción UTF-8. Pero mantener este esquema significaría que los próximos sistemas vendidos por FKC también podrían luego necesitar la migración a UTF-8 para soportar otros lenguajes. El equipo decidió entonces asignar la tarea de migrar el sistema al soporte de caracteres UTF-8 en su versión básica.

Para lograr esto debían guardarse todos los archivos con el nuevo formato que previamente estaban guardados en el formato ISO y cambiar la etiqueta \gls{HTML} que especifica el \emph{charset} en todos los archivos del sistema. Se utilizaron expresiones regulares dentro del editor de texto para modificar este parámetro satisfactoriamente.

Los otros desarrolladores realizaron las pruebas pertinentes para comprobar el resultado de la operación.

\item Implantación del SGA de FKC

El proceso de implantación del sistema consistió en modificar las tablas en el servidor objetivo y enviar los archivos a través de \gls{FTP}. Este era un proceso totalmente manual.

Para FKC debió hacerse en dos servidores distintos, uno para la filial en Munich y el otro para producción.

Se realizaron pruebas manuales para la integración del módulo con las demás partes del sistema.

\item Implantación del SGA Bibliomed

El proceso fue análogo a la implantación del SGA de FKC solo cuidando las diferencias en la estructura de la base de datos.

\end{itemize}


% section noveno_sprint (end)

\section{Décimo sprint} % (fold)
\label{sec:decimo_sprint}

\subsection{Objetivos}

\begin{itemize}
	\item Explicar a los demás desarrolladores la estructura de la solución implementada.
	\item Actualizar los esquemas de la base de datos de la empresa.
	\item Realizar del informe de pasantía.
\end{itemize}

Durante este sprint el pasante mantuvo charlas con los desarrolladores sobre el desarrollo del módulo, explicando cuales partes podían ser mejoradas, describiendo los \emph{plugins} utilizados, respondiendo preguntas y haciendo correcciones menores al sistema. 

El pasante expuso sus recomendaciones para el mejoramiento del sistema y los métodos de trabajo, así como también recibió sugerencias sobre el trabajo realizado.

\subsection{Actividades} % (fold)
\label{sub:actividades10}

\begin{itemize}

\item Actualizar los esquemas de la base de datos de la empresa

Se actualizaron los esquemas generados en la herramienta SQL Management Studio para que mostraran los cambios hechos por el pasante en los sistemas de FKC y Bibliomed.

\item Realización del informe de pasantías

El último sprint fue mayormente usado para la realización del informe de pasantías a entregar en la Universidad Simón Bolívar.

\end{itemize}


% section decimo_sprint (end)


\section{Actividades extra} % (fold)
\label{sec:actividades_extra}

Durante la realización del proyecto dada la naturaleza de la empresa como consultora informática de aprendizaje electrónico el pasante fue requerido para la realización de tareas extra de los diversos clientes de FKC.

	\subsection{Exportar preguntas de un SGA} % (fold)
	\label{sub:exportar_preguntas_de_un_sga}

	% subsection exportar_preguntas_de_un_sga (end)

	Una funcionalidad para un sistema realizado con Active Sever Pages con Visual Basic (ASP). Consintió en la implementación de una migración para los datos de uno de los clientes. Específicamente, las preguntas presentes en su base de datos. El pasante debió automatizar la generación del nuevo formato para las preguntas, marcando las distintas preguntas con sus opciones con un carácter especial marcando la respuesta correcta y separando las preguntas según su tipo (selección simple, compuesta). Luego sirviendo el resultado a través de la web en el sistema del cliente en un archivo que pudiera ser interpretado en las \emph{suites} ofimáticas comunes.

	Para la realización de esta tarea el pasante realizó la lógica enteramente dentro del lenguaje SQL, Limitando el uso de ASP al servicio de los resultados a través de la página web.

	% section actividades_extra (end)

	\subsection{Migración a UTF-8} % (fold)
	\label{sub:migracion_a_utf_8}
	
	La empresa al estar satisfecha con la migración de su SGA base a la codificación UTF-8 requirió que el pasante migrara un SGA de un cliente a esta codificación para el soporte del lenguaje turco.
	% subsection migracion_a_utf_8 (end)

	\subsection{Funcionalidad en modulo \emph{responsive}} % (fold)
	\label{sub:funcionalidad_en_modulo_responsive}
	
	El pasante apoyó a otro desarrollador en uno de sus proyectos, una herramienta para generar cursos o presentaciones dentro de uno de los SGA de un cliente.

	La herramienta permite la construcción de diapositivas con distintos elementos de interfaz básicos, como, imágenes con texto, figuras, etc. El pasante codificó los parámetros de entrada y la presentación de imágenes con leyendas al pie o encabezando las mismas, enmarcado dentro del diseño \emph{resposive}.

	% subsection funcionalidad_en_modulo_responsive (end)

	\subsection{Imágenes de las vistas de login} % (fold)
	\label{sub:imagenes_de_las_vistas_de_login}
	
	Los SGA vendidos por FKC son personalizables, entre uno de los parámetros que pueden modificarse para la adaptación a los clientes son las imágenes de la vista de inicio de sesión, y el \emph{banner} que se muestra al tope de cada vista del sistema tanto en la vista de usuario como en la vista de administrador.

	Anteriormente, los desarrolladores dentro de FKC recibían las imágenes necesarias para estas vistas y las adaptaban al sistema de manera manual y colocándolas en un fichero creado para este propósito en el servidor.

	El pasante agregó una funcionalidad para agregar estas imágenes y su color de fondo directamente desde la interfaz de super admin dentro del sistema que soporta la previsualización de la imagen escogida. Ofreciendo una manera más rápida y sencilla para la elección de dicha imagen, así como sus dimensiones y además permitiendo que los mismos clientes puedan modificar sus imágenes de inicio de sesión y \emph{banner}. 

	% subsection imagenes_de_las_vistas_de_login (end)

\section{Dificultades encontradas} % (fold)
\label{sec:dificultades_encontradas}

A lo largo de la realización del proyecto de pasantía se encontraron las siguientes dificultades:

\begin{itemize}
	\item El idioma fue una limitante para el pasante a la hora de hacer el análisis del sistema y del código a modificar en algunos casos. El código muchas veces no explicaba por si solo la funcionalidad que implementaba. Teniendo el pasante que acudir a los otros desarrolladores para la explicación de casos borde no apreciables en el análisis semántico. Además, comentarios y nombres de rutinas se encontraban escritos en alemán, idioma que el pasante no dominaba al nivel necesario para su completo entendimiento.

	\item El manejo de la internacionalización del sistema como fechas, husos horarios y lenguajes fue un reto interesante que el pasante debió sortear, siendo los clientes de diversos países dentro de la Unión Europea. 
	
	\begin{itemize}
		\item Para el manejo de fechas y husos horarios se aprovechó la interacción de los campos tipo \emph{datetime} de SQL Server y la clase \emph{DateTime} de PHP, que permitió configurar las locales dependiendo de las características del servidor donde se encuentre alojado el sistema.

		\item Para el manejo del texto en diferentes lenguajes el pasante manejó un base de datos de la empresa donde se encuentran todos los textos de la aplicación. Insertando y consultando los textos adecuados en cada ocasión. Los textos de la base de datos luego pasaban a ser parte de una función filtro de todos los textos del sistema. Dicha función detectaba el idioma guardado en la sesión HTTP y mostraba el texto oportuno. El pasante ingresó todos los textos en inglés que luego fueron traducidos por colegas de la empresa.
	\end{itemize}

\end{itemize}



% section dificultades_encontradas (end)

